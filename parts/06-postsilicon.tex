\section{Teste pós-silício}

Algumas das técnicas de verificação podem ser utilizadas na fase pré-silício
também podem ser aplicadas à fase de teste pós-silício. Como exemplo,
na~\Cref{tbl:freitas-table-i}, reproduzida a partir da Tabela I
de~\cite{Freitas:2013}, são descritos 6 \textit{checkers} com tal
característica, todos \textit{post-mortem} baseados em grafos direcionados
acíclicos que representam, por exemplo, eventos relacionados à ordem de
programa. Dentre eles, a ferramenta XCHECK~\cite{Hu:2012} foi capaz de reduzir
o custo computacional com relação a outros \textit{checkers} no estado da arte
de sua publicação. Contudo, a técnica nela aplicada, que se baseia em fazer
contínuas verificações por ciclos no grafo formado com os eventos de acesso à
memória, depende do uso de \textit{backtracking}, o que limita a escalabilidade
da ferramenta ao criar uma relação de custo exponencial com relação ao número
de processadores no sistema sob verificação, ainda que mantendo custo linear
com relação ao número de operações de memória. Conforme informado por seus
autores, porém, tal limitação não foi observada quando o número de
processadores no sistema projetado é menor ou igual a 16.

\begin{table}
    \tiny
    \centering
    \makebox[\linewidth]{%
        \hspace{4em}
        \begin{tabular}{c c c c c c c c}
            \toprule
            Autor & Ano & Uso & Tipo & Baseado em & Garantias & Monitores & \makecell{Complexidade \\ (pior caso)} \\
            \midrule
            Hangal et al. & 2004 & (Pré-) Pós-silício & Post-mortem & Inferência baseada em DAG~\footnote{\textit{Directed Acyclic Graph} --- Grafo Acíclico Direcionado} & --- & 1 & $O(n^5)$ \\
            Manovit et al. & 2005 & (Pré-) Pós-silício & Post-mortem & Inferência baseada em DAG & --- & 1 & $O(pn^3)$ \\
            Manovit et al. & 2006 & (Pré-) Pós-silício & Post-mortem & Inferência baseada em DAG & Completas & 1 & $O({(n/p)}^{p}pn^3)$ \\
            Roy et al. & 2006 & (Pré-) Pós-silício & Post-mortem & Inferência baseada em DAG & --- & 1 & $O(n^4)$ \\
            Chen et al. & 2009 & (Pré-) Pós-silício & Post-mortem & Inferência baseada em DAG & Completas & 1 & \makecell{$O(p^{3}n)$ \\ $O(p^2n^2)$} \\
            Hu et al. & 2012 & (Pré-) Pós-silício & Post-mortem & Inferência baseada em DAG & Completas & 1 & $O(p^3n)$ \\
            Shacham et al. & 2008 & Pré-silício & \textit{On-the-fly} & Um único \textit{Scoreboard} relaxado & --- & 1 & $O(p^2n^2)$ \\
            Rambo et al. & 2012 & Pré-silício & Post-mortem & Correspondência de grafo bipartido & Completas & 2 & $O(n^6/p^5)$ \\
            Freitas et al. & 2013 & Pré-silício & \textit{On-the-fly} & \textit{Scoreboards} concorrentes relaxados & Completas & 3 & $O(n^2/p)$ \\
            \bottomrule
        \end{tabular}
    }
    \caption{Reprodução (adaptada) da tabela de trabalhos correlatos em~\cite{Freitas:2013}.\label{tbl:freitas-table-i}}
\end{table}

\section{Comparação de cobertura estrutural}

As~\Cref{coverage-minus,coverage-plus} comparam as coberturas obtidas com os
quatro geradores adotados em função do número de processadores ($p$) e do
tamanho do programa de teste, expresso em número de operações ($n$). Como a
cobertura é uma propriedade dos geradores (e não dos \textit{checkers}), ela
independe do Cenário. Por isso, as~\Cref{coverage-minus,coverage-plus}
apresentam valores que valem tanto para o Cenário 1 quanto para o Cenário 2.

\begin{table}[ht]
    \tiny
    \centering
    \caption{Cobertura estrutural para PLAIN- e CHAIN- nos Cenários 1 e 2}\label{coverage-minus}
    \begin{tabular}{c|ccc|ccc}
        \toprule
               & \multicolumn{6}{c}{Cenários 1 e 2}                       \\\midrule
        n      & \multicolumn{3}{c}{PLAIN-} & \multicolumn{3}{c}{CHAIN-}  \\\midrule
        Cores  &  8      &  16     &  32    &  8      &  16     &  32     \\\midrule
        1Ki    &  0.354  &  0.393  &  0.432 &  0.358  &  0.389  &  0.418  \\\midrule
        2Ki    &  0.379  &  0.368  &  0.411 &  0.365  &  0.382  &  0.414  \\\midrule
        4Ki    &  0.396  &  0.393  &  0.484 &  0.407  &  0.407  &  0.456  \\\midrule
        8Ki    &  0.414  &  0.411  &  0.432 &  0.400  &  0.418  &  0.421  \\\midrule
        16Ki   &  0.414  &  0.460  &  0.460 &  0.400  &  0.432  &  0.442  \\
        \bottomrule
    \end{tabular}
\end{table}



Na~\Cref{coverage-minus}, a maior cobertura estrutural foi alcançada com $p =
32$ e $n = 4Ki$ tanto para PLAIN- quanto CHAIN-. De maneira geral, para $p =
\text{8 e 16}$ a cobertura foi maior quando $n = 16Ki$ para ambos os geradores,
salvo para $p = 8$ com CHAIN-. Comparando um gerador ao outro, PLAIN- conseguiu
melhores resultados levando em conta que, além de atingir um valor máximo de
cobertura maior que o de CHAIN-, alcançou uma cobertura maior em 9 dos 15 casos
quando utilizados os mesmos valores para $p$ e $n$.  Ou seja, a aplicação
apenas de \textit{Chaining} para este conjunto experimental não trouxe
benefícios para a cobertura estrutural.

\begin{table}[ht]
    \tiny
    \centering
    \caption{Cobertura estrutural para PLAIN+ e CHAIN+ nos Cenários 1 e 2\label{coverage-plus}}
    \begin{tabular}{c|ccc|ccc}
        \toprule
        & \multicolumn{6}{c}{Cenários 1 e 2} \\
        \midrule
             & \multicolumn{3}{c}{PLAIN+} & \multicolumn{3}{c}{CHAIN+} \\\midrule
        n / \textit{Cores}
             & 8     & 16    & 32         & 8     & 16    & 32         \\\midrule
        1Ki  & 0.572 & 0.586 & 0.618      & 0.604 & 0.614 & 0.625      \\\midrule
        2Ki  & 0.579 & 0.593 & 0.635      & 0.604 & 0.642 & 0.649      \\\midrule
        4Ki  & 0.572 & 0.614 & 0.656      & 0.621 & 0.656 & 0.670      \\\midrule
        8Ki  & 0.604 & 0.632 & 0.642      & 0.632 & 0.649 & 0.670      \\\midrule
        16Ki & 0.621 & 0.646 & 0.674      & 0.646 & 0.660 & 0.677      \\
        \bottomrule
    \end{tabular}
\end{table}


Na~\Cref{coverage-plus}, a maior cobertura estrutural foi alcançada com $p =
32$ e $n = 16Ki$ para ambos PLAIN+ e CHAIN+. Neste conjunto de experimentos,
foi benéfica a aplicação de \textit{Chaining}: CHAIN+ obteve melhores
resultados para todos os valores de $p$ e $n$, ainda que ambos os geradores
tenham alcançado uma cobertura muito próxima, com a máxima cobertura alcançada
por CHAIN+ sendo apenas 0.5\% a mais que a de PLAIN+.  Essa proximidade, porém,
sugere que a métrica de cobertura estrutural não é capaz de expor uma diferença
significativa no aumento da cobertura pela aplicação conjunta de
\textit{chaining} e \textit{biasing} em relação a aplicação de apenas
\textit{biasing}, uma vez que em~\cite{Andrade:2019} é demonstrado que CHAIN+
possui uma vantagem bastante significativa quando a comparação é feita pela
cobertura funcional.

De forma geral, as~\Cref{coverage-minus,coverage-plus} indicam que o aumento
tanto no número de \textit{cores} quanto no tamanho dos programas de teste é
capaz de aumentar a cobertura estrutural do gerador. Além disso, pelos
resultados expostos, o maior impactante para a cobertura estrutural dentre as
técnicas de geração foi o \textit{Biasing}, já que todos os valores de
cobertura para PLAIN+ e CHAIN+ são maiores do que os de PLAIN- e \mbox{CHAIN-,}
enquanto, conforme exposto anteriormente, apenas a aplicação de CHAIN- não foi
suficientemente benéfica. Por exemplo, em comparação com o melhor caso de
PLAIN- (48\%), o pior caso de PLAIN+ atingiu uma cobertura maior por uma razão
de 18.1\% a mais, e o melhor caso de CHAIN+ a razão foi de 39.8\% a mais.

% TODO: Descobrir o que era para completar este trecho:
% A justificativa da terceira correlação (aplicação de \textit{biasing
% constraints} para impor substituição de blocos) está no fato de que a
% ausência de

%       Possibilidades:
% Ausência de biasing de alinhamento deixa para aleatoriedade acabar causando
% replacement, enquanto que com biasing você consegue forçar o replacement e
% portanto força a cache a se mexer.

% TODO: Fazer análise em cima de transições dos níveis de cache.
% TODO: Foi mencionado de mais cópias espalhadas pela hierarquia de memória, e
%       portanto tem um grau maior de competitividade. Isso pode ser validado
%       se o número de transições de invalidação e replacements também aumenta.
%-------------------------------------------------------------------------------


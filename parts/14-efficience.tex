\section{Comparação de eficácia}

As~\Cref{minus-plain-effect,plus-plain-effect,minus-chain-effect,plus-chain-effect}
comparam, para os Cenários 1 e 2, os valores de eficácia obtidos com os quatro
geradores adotados em função do número de operações ($n$). Nessas tabelas, uma
entrada marcada em preto significa que o erro nunca foi encontrado por teste
algum.

\begin{table}[ht]
    \tiny
    \centering
    \caption{Estimativa de eficácia para PLAIN-\label{minus-plain-effect}}
    \begin{tabular}{c|ccccc|ccccc}
        \toprule
        & \multicolumn{5}{c}{Cenário 1} & \multicolumn{5}{c}{Cenário 2} \\
        \midrule
        n    &  F1               &  F2    &  F3    &  F4    &  f29              &  F1               &  F2    &  F3    &  F4    &  f29             \\
        \midrule
        1Ki  &  \notfound{}0.00  &  \textbf{0.13}  &  0.02  &  0.01  &  \notfound{}0.00  &  \notfound{}0.00  &  \textbf{0.04}  &  0.01  &  0.00  &  \notfound{}0.00 \\
        2Ki  &  \notfound{}0.00  &  \textbf{0.22}  &  0.04  &  0.02  &  \notfound{}0.00  &  \notfound{}0.00  &  \textbf{0.06}  &  0.02  &  0.01  &  \notfound{}0.00 \\
        4Ki  &  0.01             &  \textbf{0.39}  &  0.09  &  0.06  &  \notfound{}0.00  &  \notfound{}0.00  &  \textbf{0.13}  &  0.05  &  0.05  &  \notfound{}0.00 \\
        8Ki  &  0.02             &  \textbf{0.63}  &  0.12  &  0.07  &  \notfound{}0.00  &  0.01             &  \textbf{0.27}  &  0.04  &  0.05  &  \notfound{}0.00 \\
        16Ki &  \notfound{}0.00  &  \textbf{0.78}  &  0.13  &  0.15  &  \notfound{}0.00  &  \notfound{}0.00  &  \textbf{0.37}  &  0.09  &  0.14  &  \notfound{}0.00 \\
        \bottomrule
    \end{tabular}
\end{table}


Na~\Cref{minus-plain-effect}, em ambos os cenários tanto F1 quanto f29 foram os
mais difíceis de se expor, principalmente f29 que não foi exposto em qualquer
um dos cenários. F2, F3 e F4, porém, foram detectados em todos os cenários,
sendo F2 o de mais fácil exposição. Ao se passar do Cenário 1 para o Cenário 2,
a eficácia de todos os erros é reduzida para todos os valores de $n$. A de F2,
por exemplo, é reduzida cerca de 1/3 ao se trocar o cenário.

\begin{table}[ht]
    \tiny
    \centering
    \caption{Estimativa de eficácia para CHAIN-\label{minus-chain-effect}}
    \begin{tabular}{c|ccccc|ccccc}
        \toprule
        & \multicolumn{5}{c}{Cenário 1} & \multicolumn{5}{c}{Cenário 2} \\
        \midrule
% PRSB
        n        & F1   & F2            & F3   & F4   & f29             & F1              & F2   & F3   & F4   & f29              \\
        \midrule
        1Ki      & 0.02 & \textbf{0.13} & 0.16 & 0.04 & \notfound{}0.00 & \notfound{}0.00 & \textbf{0.03} & 0.06 & 0.02 & \notfound{}0.00  \\
        2Ki      & 0.03 & \textbf{0.19} & 0.23 & 0.03 & \notfound{}0.00 & 0.01            & \textbf{0.06} & 0.18 & 0.02 & \notfound{}0.00  \\
        4Ki      & 0.02 & \textbf{0.34} & 0.33 & 0.12 & \notfound{}0.00 & \notfound{}0.00 & \textbf{0.06} & 0.27 & 0.10 & \notfound{}0.00  \\
        8Ki      & 0.06 & \textbf{0.54} & 0.48 & 0.12 & \notfound{}0.00 & 0.01            & \textbf{0.17} & 0.42 & 0.10 & \notfound{}0.00  \\
        16Ki     & 0.04 & \textbf{0.76} & 0.67 & 0.19 & \notfound{}0.00 & 0.01            & \textbf{0.28} & 0.59 & 0.17 & \notfound{}0.00  \\
        \bottomrule
    \end{tabular}
\end{table}


Na~\Cref{minus-chain-effect}, CHAIN- é capaz de expor F1 para todos os tamanhos
de programa apenas no Cenário 1, mas não nos casos em que $n = 1Ki$ e $n = 4Ki$
do Cenário 2. Assim como com PLAIN-, f29 não é exposto em nenhum dos valores de
$n$ para ambos os cenários, o que resulta no aumento do esforço observado
na~\Cref{both-minus-effort} ao se aumentar o tamanho do programa.  Ao se passar
do Cenário 1 para o Cenário 2, também como em PLAIN-, a eficácia para todos os
erros foi reduzida, com F1 não sendo exposto para dois valores de $n$.

\begin{table}[ht]
    \tiny
    \centering
    \caption{Estimativa de eficácia para PLAIN+\label{plus-plain-effect}}
    \begin{tabular}{c|ccccc|ccccc}
        \toprule
        & \multicolumn{5}{c}{Cenário 1} & \multicolumn{5}{c}{Cenário 2} \\
        \midrule
        n        & F1   & F2            & F3   & F4   & f29           & F1   & F2            & F3   & F4   & f29  \\
        \midrule
        1Ki      & 0.18 & \textbf{0.73} & 0.14 & 1.00 & \textbf{0.99} & 0.06 & \textbf{0.06} & 0.90 & 1.00 & \textbf{0.24} \\
        2Ki      & 0.11 & \textbf{0.88} & 0.16 & 1.00 & \textbf{1.00} & 0.03 & \textbf{0.05} & 0.96 & 1.00 & \textbf{0.42} \\
        4Ki      & 0.26 & \textbf{0.96} & 0.26 & 1.00 & \textbf{1.00} & 0.09 & \textbf{0.09} & 0.93 & 1.00 & \textbf{0.61} \\
        8Ki      & 0.31 & \textbf{0.98} & 0.30 & 1.00 & \textbf{1.00} & 0.12 & \textbf{0.12} & 0.97 & 1.00 & \textbf{0.78} \\
        16Ki     & 0.31 & \textbf{0.99} & 0.32 & 1.00 & \textbf{1.00} & 0.13 & \textbf{0.12} & 0.98 & 1.00 & \textbf{0.89} \\
        \bottomrule
    \end{tabular}
\end{table}


Na~\Cref{plus-plain-effect}, F4 é o único garantidamente exposto (1.00 de
eficácia) independentemente do Cenário e do valor de $n$, o que significa que
programas de 1Ki são o suficiente para sua exposição. Levando em conta o
Cenário 1 e a~\Cref{both-plus-effort}, apesar de PLAIN+ conseguir elevar a
eficácia para alguns erros (F1, F2, F3, f29), em alguns casos o ganho não foi o
suficiente para reduzir o esforço de verificação conforme se aumentou o tamanho
dos programas de teste (F1, F3).

\begin{table}[ht]
    \tiny
    \centering
    \caption{Estimativa de eficácia para CHAIN+\label{plus-chain-effect}}
    \begin{tabular}{c|ccccc|ccccc}
        \toprule
        & \multicolumn{5}{c}{Cenário 1} & \multicolumn{5}{c}{Cenário 2} \\
        \midrule
        n        & F1            & F2            & F3   & F4   & f29           & F1            & F2            & F3   & F4   & f29  \\
        \midrule
        1Ki      & \textbf{0.84} & \textbf{0.66} & 0.72 & 1.00 & \textbf{0.99} & \textbf{0.41} & \textbf{0.38} & 0.86 & 1.00 & \textbf{0.22} \\
        2Ki      & \textbf{0.74} & \textbf{0.79} & 0.86 & 1.00 & \textbf{1.00} & \textbf{0.48} & \textbf{0.53} & 0.96 & 1.00 & \textbf{0.31} \\
        4Ki      & \textbf{0.89} & \textbf{0.93} & 0.89 & 1.00 & \textbf{1.00} & \textbf{0.58} & \textbf{0.61} & 0.96 & 1.00 & \textbf{0.47} \\
        8Ki      & \textbf{0.97} & \textbf{0.95} & 0.98 & 1.00 & \textbf{1.00} & \textbf{0.76} & \textbf{0.68} & 0.99 & 1.00 & \textbf{0.64} \\
        16Ki     & 0.94          & 0.97          & 0.98 & 1.00 & \textbf{1.00} & 0.81          & 0.83          & 0.99 & 1.00 & \textbf{0.77} \\
        \bottomrule
    \end{tabular}
\end{table}


Na~\Cref{plus-chain-effect}, como na~\Cref{plus-plain-effect}, F4 foi o único
garantidamente exposto independentemente do Cenário e do valor de $n$. A
eficácia do Cenário 2 para os demais erros foi reduzida com relação ao Cenário
1, com o valor mais impactante sendo o de f29 que, para $n = 1Ki$, sofreu uma
perda de 77\% de eficácia, e para os demais valores de $n$ deixou de ser
garantidamente exposto. Aumentando, porém, o tamanho dos programas, a eficácia
para f29 aumentou por uma razão de 3.5, chegando próximo de 100\% de eficácia.
Vale também notar que, conforme reportado pelos autores das técnicas
\textit{Biasing} e \textit{Chaining}~\cite{Andrade:2019}, apenas o uso de
\textit{Chaining} não foi o suficiente para trazer melhorias significativas,
afinal o maior ganho de eficácia se deu no uso de ambas as técnicas
simultaneamente (CHAIN+), com o maior impacto advindo da \textit{Biasing} (como
pode ser observado comparando os resultados de PLAIN+ e CHAIN- com PLAIN-).

O conjunto
das~\Cref{minus-plain-effect,minus-chain-effect,plus-plain-effect,plus-chain-effect}
indica que, por mais que um gerador seja capaz de elevar sua eficácia na
exposição de um erro de projeto, a observabilidade ainda é um limitante, vide a
degradação média de 57\% na eficácia para os casos mais impactantes (destacados
nas tabelas anteriores) para vários erros de projeto (F1, F2, F4, f29) ao se
passar do Cenário 1 para o Cenário 2.  O único erro de projeto em que esse
efeito não pôde ser observado foi F3, que teve maior eficácia no Cenário 2
quando aplicado \textit{Biasing}.  Apesar disso, para programas de $n = 16Ki$,
CHAIN+ --- que teve, no geral, a maior eficácia dentre os geradores ---
manteve, no Cenário 2, valores de eficácia próximos do Cenário 1 e
significativamente maiores que os de PLAIN- --- que, ao contrário de CHAIN+,
teve erros que não foi capaz de expor ---, o que indica que a aplicação
combinada de \textit{Biasing} e \textit{Chaining} é capaz de compensar o
impacto da troca de Cenário, e logo a imposição de restrições nos programas de
teste podem ser benéficas para ambos os Cenários. Isso também indica, ainda,
que é possível explorar meios independentes do Cenário ou \textit{checker}
(como é o caso dos geradores) a fim de melhorar a eficácia e reduzir o impacto
de limitações na observabilidade.

%%-------------------------------------------------------------------------------
%% For LC
%\review{%
%Essa diferença nos resultados quanto à presença de \textit{biasing constraint}
%tem relação com a probabilidade de um bloco ser substituído na \textit{cache}:
%como nas técnicas PLAIN+ e CHAIN+ o \textit{true sharing} é imposto, há maior
%estímulo das transições de sicronização. Além disso, o aumento no tamanho do
%programa de teste acaba por possibilitar a geração de
%}
%
%% TODO: Criar tabela (baseada na Tabela 4) com separação entre número de cores
%%       para F4.
%%-------------------------------------------------------------------------------

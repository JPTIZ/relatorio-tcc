\section{Infraestrutura}

\newcommand{\aboutcheckerrevision}{\footnote{%
    Foi utilizada a mesma versão do simulador em~\cite{Andrade:2017}, uma
    revisão baseada em \url{http://repo.gem5.org/gem5/rev/54d3ef2009a2} com
    modificações tais como a adição de erros artificiais nos controladores de
    memória \textit{cache}.
}}

% TODO: Descrever relevância/impacto dessas opções.
Para realização dos testes, foi utilizado como ferramenta de simulação o
\textit{gem5}~\cite{Binkert:2011:GS:2024716.2024718}\aboutcheckerrevision%
seguindo as mesmas configurações de~\cite{Andrade:2017}: modelo de temporização
fora de ordem (O3), modo \textit{system call emulation}, modelo \textit{Ruby}
para o subsistema de memória e rede de interconecção \textit{simple}. Para
arquitetura alvo, manteve-se o modelo de consistência Alpha e ISA SPARC\@.

% TODO: Corrigir depois de inserir novos erros.
Os erros artificiais inseridos no sistema de memória também foram os mesmos
usados em~\cite{Andrade:2017}, denominados F1, F2, F3, F4 e f29. A descrição
desses erros se encontra na Tabela~\ref{errors-desc} (adaptada da Tabela 1 da
mesma referência) e é dada em termos da máquina de estados do controlador de
\textit{cache} na hierarquia de memória. Os resultados experimentais associados
a F1 e f29 para o MSB já foram apresentados na literatura~\cite{Andrade:2019}
como D8 e D5, respectivamente, mas são reproduzidos no~\Cref{chp:results} desta
monografia para referência.

Os testes e verificações foram executados em uma estação de trabalho com um
processador Intel Xeon E5430 2.66GHz e 4GB de RAM\@.

\begin{table*}[hp]
    \centering
    \begin{adjustbox}{max width=\columnwidth}
        \begin{tabular}{cccccc}
            \toprule%
            \textbf{ID} & \textbf{Localização} & \textbf{Estado atual} & \textbf{Evento de entrada} & \textbf{Próximo estado} & \textbf{Ação de saída impedida} \\
            \midrule%
            F1  & L1 & E\_IL0 & L0\_DataAck & MM & u\_writeDataFromL0Response \\
            F2  & L1 & M\_IL0 & WriteBack & MM\_IL0 & u\_writeDataFromL0Request \\
            F3  & L0 & E & Store & E em vez de M & --- \\
            F4  & L1 & IS & Data\_Exclusive & E & u\_writeDataFromL2Response \\
            f29 & L1 & M\_IL0 & L0\_DataAck & EE em vez de MM & --- \\
            \bottomrule
        \end{tabular}
    \end{adjustbox}
    \caption{Descrição das classes de erros artificais, adaptada
             de~\cite{Andrade:2017}\label{errors-desc}}
\end{table*}


\section{Objetivo e Escopo}

Três ferramentas principais são necessárias para viabilizar a
\textbf{verificação pré-silício} do subsistema de memória compartilhada:

\begin{itemize}
    \item Um \textbf{simulador} (que represente o comportamento do sistema de
        memória sob projeto);
    \item Um \textbf{\textit{checker}} (capaz de indicar um comportamento
        incorreto e parar o simulador assim que um erro é detectado); e
    \item Um \textbf{gerador} automático de programas de teste (capaz de
        fornecer estímulos capazes de induzir no sistema os comportamentos
        esperados).
\end{itemize}

Como a observabilidade de uma representação do projeto é limitada, eventos de
leitura e escrita podem ser monitorados em diferentes pontos no domínio de cada
núcleo (múltiplos monitores por núcleo), o que tende a aumentar a eficácia e a
eficiência na detecção de erros de projeto.

Embora o \textbf{teste pós-silício} do subsistema de memória também use um
\textbf{gerador} automático, ele requer a execução de teste diretamente num
\textbf{protótipo} do \textit{multicore chip}. Como a observabilidade de um
\textit{chip} é limitada, durante a execução de um teste, monitora-se a
interface de cada núcleo com a memória para capturar eventos de leitura e
escrita (um único monitor no domínio de cada núcleo). A sequência de eventos
monitorados ao longo do tempo em cada núcleo é denominada de \textit{trace}.
Terminada a execução de um teste, os \textit{traces} monitorados em cada um dos
núcleos são disponibilizados para um \textbf{\textit{checker}} que analisa se
as sequências de eventos satisfazem o comportamento esperado do sistema.
Entretanto, a observabilidade limitada do \textit{chip} tende a comprometer a
cobertura, a eficácia e a eficiência na detecção de erros.

Esta monografia avalia o \textbf{impacto da observabilidade} na eficácia e na
eficiência de geradores exporem erros de projeto, tendo como maior contribuição
a comparação entre os resultados experimentais obtidos com os \textit{checkers}
MSB~\cite{Freitas:2013} e XCHECK~\cite{Hu:2012}, descritos no~\Cref{chp:tools}
deste trabalho, quando os programas de teste são gerados com o uso das técnicas
\textit{Chaining} e \textit{Biasing}~\cite{Andrade:2019}. Para isso, os
programas de teste sintetizados por um gerador são aplicados a dois cenários
distintos:

\begin{itemize}
    \item Cenário múltiplos monitores por núcleo; e
    \item Cenário um único monitor por núcleo.
\end{itemize}

Como o primeiro cenário é típico da verificação pré-silício e o segundo é
típico do teste pós-silício, a avaliação aqui proposta deve fornecer uma
estimativa da viabilidade de uso de geradores que foram originalmente
desenvolvidos para o primeiro cenário se fossem usados no segundo cenário.

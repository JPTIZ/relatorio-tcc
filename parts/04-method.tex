\section{Método de Pesquisa}

O método adotado consiste na realização de experimentos sobre uma
\textbf{representação de projeto do sistema sob verificação}, ou seja, na
execução de testes através da simulação daquela representação, utilizando como
infraestrutura um \textbf{simulador} de domínio público denominado
gem5~\cite{Binkert:2011:GS:2024716.2024718}. A representação de projeto
utilizada para o subsistema de memória corresponde ao módulo \textit{Ruby}
daquele simulador, a qual permite a descrição das máquinas de estado dos
protocolos de coerência.

O mesmo simulador será utilizado em ambos os cenários. No Cenário 1, ele fará o
papel de uma descrição em HDL do projeto de um \textit{multicore chip}. No
Cenário 2, ele fará o papel do protótipo do \textit{multicore chip} (que não
pode ser produzido no ambiente acadêmico).

Para representar um \textit{checker} típico no Cenário 1 (pré-silício), será
utilizada uma técnica de análise automática dos eventos registrados durante a
simulação, a qual foi desenvolvida localmente~\cite{Freitas:2013} e cujo código
está portanto disponível para uso no laboratório hospedeiro. Para representar
um \textit{checker} típico no Cenário 2 (pós-silício), será utilizada uma
implementação do \textit{checker} XCHECK~\cite{Hu:2012}, resultado da adaptação
feita do \textit{checker} original em trabalho desenvolvido anteriormente no
grupo de pesquisa~\cite{Andrade:2016}.

Quatro geradores de teste aleatórios serão avaliados quanto à sensibilidade ao
impacto da limitação de observabilidade. Um deles (denominado PLAIN-) usa
restrições convencionais~\cite{Rambo:2011}, número de operações e de variáveis
compartilhadas. Os demais exploram restrições não convencionais
adicionais~\cite{Andrade:2017}: um explora restrições sobre o espaço de
endereçamento (denominado PLAIN+), outro impõe padrões para a geração de
cadeias de operações entre \textit{threads} distintas (denominado CHAIN-) e um
último explora ambos os tipos de restrições (CHAIN+).

Os quatro geradores serão comparados ao se usar a mesma representação de
projeto e o mesmo simulador. Apenas os \textit{checkers} serão distintos nos
Cenários 1 e 2 para caracterizarem a observabilidade típica das diferentes
fases do projeto.

Para a simulação de erros de projeto, serão utilizados como base os cinco erros
artificiais descritos em~\cite{Andrade:2017}.

Para a comparação serão adotadas as seguintes métricas: cobertura estrutural,
esforço computacional e probabilidade de detecção de erros.

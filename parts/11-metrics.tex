\section{Métricas}

Foram utilizadas três métricas para avaliar as técnicas de verificação
enunciadas nos diferentes cenários:

\begin{description}
    \item[Cobertura estrutural:] A razão entre número de transições da máquina
        de estados do controlador de cada \textit{cache} que foram efetivamente
        cobertas pelo teste e o número total de transições dessa mesma máquina;

    \item[Eficácia:] A razão entre o número de execuções de programas de teste
        em que um determinado erro de projeto foi exposto e o número total de
        execuções de programas de teste no sistema com tal erro, servindo para
        estimar a probabilidade de exposição de um erro para um determinado
        conjunto de parâmetros de execução;

    \item[Esforço:] O tempo necessário (em média) para que um determinado erro
        de projeto seja exposto. Diz-se que um erro foi \textbf{exposto} em um
        determinado cenário quando o \textit{checker} utilizado no cenário é
        capaz de detectar a respectiva violação do modelo de memória. A
        equação~\ref{effort-eq}, retirada de~\cite{Andrade:2017}, descreve a
        medida de esforço ($EF$), em que $T$ representa o conjunto de programas
        de teste gerados por um gerador $G$ (e portanto $|T|$ representa o
        número de testes executados), $\epsilon$ representa a eficácia de $G$
        para o conjunto $T$ e $\hat{t}$ a média do tempo de execução dos
        testes.
\end{description}

\begin{equation}
    \label{effort-eq}
    EF_G = \begin{cases}
               \hat{t}/\epsilon, & \mbox{se } \epsilon \neq 0\\
               |T|.\hat{t},      & \mbox{se } \epsilon = 0
           \end{cases}
\end{equation}

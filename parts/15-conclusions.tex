%\remove{%
%    Pelos resultados expostos, é possível verificar que, de maneira geral, o
%    uso de cadeias de dependências feitas por CHAIN acaba por melhorar a
%    eficácia da exposição de erros, o que acaba por também reduzir o esforço
%    necessário para tal. Porém, essa correlação nem sempre se mantém
%    verdadeira, ou seja: é possível que o tempo necessário para executar um
%    programa de teste supere o ganho com eficácia. Porém, como foi o caso de F1
%    para PLAIN-, tal aumento pode se mostrar necessário para que se possa expor
%    um determinado erro.
%}
%
%\remove{%
%    Também é possível notar que a imposição de \textit{biasing constraints} no
%    programa de teste (como é o caso de PLAIN+ e CHAIN+) leva a uma redução do
%    esforço e uma elevação na eficácia, com ressalva de um único caso dentre os
%    testes realizados em que houve redução da eficácia, mas manteve-se a
%    redução no esforço. Tais \textit{constraints} foram capazes, inclusive, de
%    possibilitar a exposição de erros que, independente do gerador ou
%    \textit{checker}, não eram possíveis de se detectar mesmo com o aumento do
%    tamanho do programa de teste, como foi o caso de f29.
%}
%
%\remove{%
%    Além disso, é notável o fato de que o aumento no tamanho dos programas de
%    teste, apesar de melhorar a eficácia dos geradores em vários casos, acaba
%    por aumentar também o esforço necessário quando há a imposição de
%    restrições no programa de teste. Ou seja, aumentar o tamanho do programa de
%    teste para esse caso faz com que um erro de projeto seja exposto com maior
%    frequência, porém levando mais tempo quando se consegue a devida exposição.
%}
%
%\remove{%
%    Sendo assim, impor restrições para geração de programas de teste, em
%    conjunto com o aproveitamento de sequências conhecidas de operações
%    essenciais para sistema de memória, pode auxiliar na verificação e teste
%    dos \textit{Multicore-Chips} em questão, expondo erros com maior frequência
%    em menos tempo. Também é visível que a maior observabilidade de um cenário
%    pré-silício pode ser explorada de forma a trazer resultados positivos,
%    visto que para boa parte dos casos o \textit{checker} do Cenário 1 foi
%    capaz de expor erros de projeto com frequência maior ou igual à do Cenário
%    2 para as mesmas condições.
%}

A métrica de eficácia permitiu quantificar o impacto da redução da
observabilidade do Cenário 2 (pós-silício) comparada ao Cenário 1 (pré-silício)
para cada gerador. Embora a degradação da eficácia varie com o gerador
utilizado, para os erros que foram sempre encontrados, a mínima degradação foi
de 6.7\% (para o gerador PLAIN-) e máxima de 94\% (para o gerador PLAIN+).

Os geradores normalmente utilizados para teste pré-silício assemelham-se aos
geradores PLAIN-, pois não utilizam as restrições não convencionais propostas
nas técnicas \textit{Chaining} e \textit{Biasing}~\cite{Andrade:2019}. Apesar
de ter a mínima degradação, a eficácia de PLAIN- já era, no geral, a menor para
o Cenário 1 dentre os quatro geradores, tendo a mudança para o Cenário 2 apenas
a reduzindo ainda mais com uma degradação máxima de 73\%. Como agravante, se
forem considerados os erros que deixaram de ser encontrados no Cenário 2,
pode-se dizer que PLAIN- teve uma degradação máxima de 100\%.

Por outro lado, para um gerador que explorasse \textit{Chaining} e
\textit{Biasing} simultaneamente, ou seja, que se assemelhasse a CHAIN+, a
degradação na eficácia seria bastante perceptível para programas pequenos (para
programas de tamanho 1Ki, uma degradação mínima de 42\% e máxima de 77\%).
Porém, a degradação é reduzida em programas maiores (para programas de tamanho
16Ki, uma degradação mínima de 13.8\% e máxima de 22.9\%).

% Frase final da monografia:
Apesar de haver uma degradação na eficácia no Cenário 2, a adoção combinada das
técnicas \textit{Biasing} e \textit{Chaining} (CHAIN+) a compensa com um ganho
substancial de eficácia em relação a um gerador convencional (PLAIN-). Além
disso, conforme já demonstrado pelos autores dessas
técnicas~\cite{Andrade:2019}, há também um ganho significativo na cobertura,
que como explicado anteriormente independe do Cenário. Assim, é seguro concluir
que, mesmo sendo projetadas para a verificação pré-silício, tais técnicas podem
ser aplicadas no contexto de teste pós-silício de modo a compensar suas
limitações de observabilidade.

%Como a cobertura não se altera entre os cenários pré e pós-silício e os
%autores das técnicas \textit{Chaining} e
%\textit{Biasing}~\cite{Andrade:2019} já demonstraram sua cobertura
%significativamente maior (CHAIN+) em relação ao gerador convencional
%(PLAIN-), ao adotá-las para teste pós-silício, a maior cobertura acaba
%compensando parcialmente a redução de observablidade, mantendo uma eficácia
%maior do que quando na ausência dessas técnicas. Portanto, conclui-se que
%aquelas técnicas, desenvolvidas para verificação pré-silício, podem ser
%aplicadas com substancial vantagem no contexto de teste pós-silício de
%multicore chips.

% Como a cobertura não se altera entre os cenários pré e pós-silício e os
% autores das técnicas \textit{Chaining} e
% \textit{Biasing}~\cite{Andrade:2019} já demonstraram sua cobertura
% significativamente maior (CHAIN+) em relação ao gerador convencional
% (PLAIN-), ao adotá-las para teste pós-silício, a maior cobertura acaba
% compensando parcialmente a redução de observablidade, diminuindo assim a
% degradação da eficácia na detecção de erros. Portanto, conclui-se que
% aquelas técnicas, desenvolvidas para verificação pré-silício, podem ser
% aplicadas com substancial vantagem no contexto de teste pós-silício de
% multicore chips.

\section{Geradores}

Conforme descrito em~\cite{Andrade:2017}, os geradores de programas de teste
possuem parâmetros que visam reforçar algum aspecto da verificação.

Há parâmetros comuns aos geradores utilizados neste trabalho, sendo: o número
de \textit{threads}, o de operações de memória, o de endereços compartilhados,
e uma semente para pseudo-aleatoriedade. PLAIN- não impõe restrições adicionais
sobre a geração de instruções, selecionando aleatoriamente as instruções de uma
\textit{thread} sem levar em conta quais já foram selecionadas, nem sobre a
atribuição de endereços.

PLAIN+ mantém o mesmo gerador de instruções aleatórias que PLAIN-, porém
substitui a atribuição de endereços pela técnica \textit{Biasing}, enquanto
CHAIN- mantém o mesmo sistema de atribuição de endereços de PLAIN-, porém
substitui o gerador de instruções aleatórias pela aplicação de
\textit{Chaining}. \textit{Biasing} (Seção VI de~\cite{Andrade:2019}) impõe
restrições sob o espaço de endereçamento buscando controlar o grau de
competitividade entre endereços compartilhados, e \textit{Chaining} (Seção V
de~\cite{Andrade:2019}) restringe as sequências de instruções geradas focando
em fazer com que mais eventos de memória conflitem em um mesmo local
compartilhado a fim de aumentar a cobertura de transições do controlador de
\textit{cache}.

Por fim, CHAIN+ aplica ambas \textit{Chaining} e \textit{Biasing}, sendo então
o gerador mais restritivo dentre os quatro.

% Descrição das ferramentas

No contexto de verificação e teste de \textit{multicore chips}, dois tipos de
ferramentas podem ser utilizados pelo projetista do sistema de memória:
\textbf{Geradores} e \textbf{\textit{Checkers}}. Um Gerador é responsável por
sintetizar programas de teste paralelos que serão executados --- via simulação
(pré-silício) ou através da execução no \textit{hardware} da arquitetura-alvo
(pós-silício) --- e seus eventos com relação a acessos à memória registrados,
gerando \textit{traces}.

Já um \textit{checker} é responsável por analisar o conjunto de \textit{traces}
disponibilizado ao executar os programas de teste. Através dessa análise, um
\textit{checker} é capaz de apontar a existência de falhas no sistema de
memória implementado.

A seguir, serão descritos os \textit{Checkers} e Geradores utilizados nesta
monografia.

\begin{resumo}
    Com o aumento do número de \textit{cores} de um \textit{chip}, o
    \textit{hardware} que gerencia o sistema de memória compartilhada torna-se
    cada vez mais complexo e suscetível a erros de projeto. Por isso é
    importante testar se o projeto do \textit{hardware} está correto, para
    evitar que erros acabem se propagando para o \textit{chip}, sendo
    detectados apenas no protótipo fabricado. Por isso, a verificação
    pré-silício do sistema de memória compartilhada é crucial para a redução de
    custos de desenvolvimento. Como a verificação pré-silício trata de executar
    programas de teste em um simulador, é possível analisar a consistência de
    um sistema de memória com maior liberdade do que em um ambiente de teste
    pós-silício, já que há mais pontos acessíveis para se registrar eventos
    relacionados ao acesso à memória. Este trabalho avalia experimentalmente o
    impacto da redução da observabilidade do sistema de memória na eficácia, na
    cobertura e no esforço requerido para encontrar erros de projeto. Ele
    compara os mesmos geradores de programas de teste quando utilizados na
    verificação pré-silício e no teste pós-silício de um \textit{multicore
    chip}.

    \vspace{\onelineskip}
    \noindent
    \textbf{Palavras-chave}: \listaassuntos%
\end{resumo}

\afterpage{\null\newpage}

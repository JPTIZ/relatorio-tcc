\section{Comparação de esforço}

As~\Cref{both-minus-effort,both-plus-effort} comparam os geradores quanto ao
esforço de verificação quando programas de teste gerados por eles são enviados
ao \textit{checker} do Cenário 1. Os valores foram medidos para cada erro de
projeto (F1, F2, F3, F4 e f29) separadamente e estão expostos em função do
tamanho do programa de teste ($n$). Medidas de esforço para o \textit{checker}
do Cenário 2 não foram apresentadas pois, nesse caso, não trariam valores
realistas: o esforço é baseado no tempo de execução e, por não ser viável para
este trabalho sintetizar o \textit{multicore-chip} --- como é característica do
cenário pós-silício ---, o \textit{checker} em questão executaria em um
ambiente simulado.

\begin{table}[ht]
    \tiny
    \centering
    \caption{Estimativa de esforço para PLAIN- e CHAIN- no Cenário 1}\label{both-minus-effort}
    \begin{tabular}{c|ccccc|ccccc}
        \toprule
        & \multicolumn{5}{c}{PLAIN-} & \multicolumn{5}{c}{CHAIN-} \\
        \midrule
        n    &  F1       &  F2      &  F3      &  F4       &  f29       &  F1       &  F2      &  F3      &  F4      &  f29       \\\midrule
        1Ki  &  1297.29  &  161.98  &  648.54  &  2263.63  &  12614.27  &  1242.25  &  207.00  &  113.03  &  280.72  &  12208.55  \\\midrule
        2Ki  &  1510.45  &  150.53  &  377.85  &  667.96   &  14219.17  &  699.19   &  126.93  &  43.59   &  326.76  &  13510.66  \\\midrule
        4Ki  &  1959.86  &  81.04   &  217.30  &  182.15   &  17263.38  &  1680.25  &  152.06  &  33.90   &  84.83   &  15679.00  \\\midrule
        8Ki  &  2835.04  &  56.98   &  354.39  &  253.48   &  23329.20  &  1122.19  &  74.00   &  28.43   &  103.25  &  20045.94  \\\midrule
        16Ki &  4577.49  &  66.49   &  284.81  &  135.36   &  35045.80  &  1693.03  &  65.75   &  28.97   &  88.86   &  28974.76  \\
        \bottomrule
    \end{tabular}
\end{table}


Na~\Cref{both-minus-effort}, é possível observar que erro f29 demandou o maior
esforço com ambos os geradores PLAIN- e CHAIN-. Para ambos os geradores, ao se
aumentar o valor de $n$, alguns erros tiveram o esforço de verificação reduzido
(F2, F3, F4) e nos demais o esforço foi elevado (F1 e f29). Para $n = 1Ki$, F2
foi o único erro o qual CHAIN- necessitou de mais esforço do que PLAIN-. Porém,
aumentando o valor de $n$ para $16Ki$, ainda que o esforço para F2 tenha
diminuído para ambos os geradores, CHAIN- foi capaz de reduzir o esforço de
forma que todos os erros demandassem menos esforço do que com PLAIN-. Quanto a
f29, para $n = 1Ki$ a $n = 16Ki$, a razão entre os esforços de CHAIN- e PLAIN-
passou de 0.97 para 0.83. Isso significa que para o esforço, dadas estas
condições experimentais, a aplicação da técnica \textit{Chaining} traz melhores
resultados do que o uso de restrições convencionais conforme se aumenta o
tamanho do programa de teste, ainda que não necessariamente reduza o esforço.

\begin{table}[ht]
    \tiny
    \centering
    \caption{Estimativa de esforço para PLAIN+ e CHAIN+ no Cenário 1}\label{both-plus-effort}
    \begin{tabular}{c|ccccc|ccccc}
        \toprule
        & \multicolumn{5}{c}{PLAIN+} & \multicolumn{5}{c}{CHAIN+} \\
        \midrule
        n    & F1    & F2    & F3    & F4   & f29   & F1    & F2    & F3    & F4   & f29  \\\midrule
        1Ki  & 48.58 & 9.21  & 50.50 & 5.01 & 5.34  & 7.14  & 9.90  & 7.15  & 5.05 & 5.25 \\\midrule
        2Ki  & 73.36 & 7.77  & 68.58 & 5.11 & 5.20  & 6.91  & 8.76  & 6.85  & 5.15 & 5.19 \\\midrule
        4Ki  & 53.35 & 6.78  & 51.87 & 5.13 & 5.37  & 7.65  & 7.22  & 7.47  & 5.13 & 5.28 \\\midrule
        8Ki  & 57.16 & 6.90  & 59.98 & 5.19 & 5.27  & 6.44  & 7.95  & 6.09  & 5.18 & 5.23 \\\midrule
        16Ki & 85.41 & 6.89  & 86.36 & 5.26 & 5.56  & 8.58  & 8.47  & 7.38  & 5.28 & 5.35 \\
        \bottomrule
    \end{tabular}
\end{table}


Na~\Cref{both-plus-effort}, com exceção de F1 e F3, todos os valores de esforço
foram semelhantes entre os geradores. No caso de F1 e F3, os valores de esforço
com CHAIN+ foi decisivamente menor do que os de PLAIN+, tendo PLAIN+
necessitado de 11.7 vezes o esforço de CHAIN+ para F3 quando $n = 16Ki$.

Comparando as~\Cref{both-minus-effort,both-plus-effort} é possível observar
que, assim como na cobertura estrutural, o maior impacto está no uso da técnica
\textit{Biasing}, ainda que os melhores resultados foram com a combinação dela
com \textit{Chaining}, corroborando com os resultados obtidos
em~\cite{Andrade:2019}. Tomando os valores para F3 e f29 (menor e maior) como
exemplo, o esforço com CHAIN+ foi de 0.25\% do valor de esforço com PLAIN+ para
F3 e menos de 0.02\% para f29.

%----------------------------------------------------------------------
% Comandos criados pelo usuário
%----------------------------------------------------------------------
\renewcommand{\bf}[1]{\textbf{#1}}
\newcommand{\review}[1]{{\color{green!65!black}{#1}}}
\newcommand{\remove}[1]{{\sout{#1}}}
\newcommand{\newcontent}[1]{{\color{blue}{#1}}}
\newcommand{\tolc}[1]{{\color{red}{#1}}}
\newcommand{\critical}[1]{{\color{red}\textbf{{#1}}}}
\newcommand{\verycritical}[1]{{\color{red}\textbf{\uppercase{{#1}}}}}

\newcommand{\blue}[1]{{\color{blue}{#1}}}
\newcommand{\green}[1]{{\color{green!65!black}{#1}}}

% For to-do listing
\newcommand{\todo}[1]{\textit{\color{red}{#1}}}

\newenvironment{todolist}
{%
    \begin{itemize}
        \color{red}
}{%
    \end{itemize}
}

\newcommand{\todofootnote}[2]{\todo{#1}\footnote{\todo{#2}}}

% Regras de coloração de tabela

\definecolor{shadecolor}{rgb}{0.8,0.8,0.8}
\newcommand\VRule[1][\arrayrulewidth]{\vrule width #1}

\newcommand{\shadecell}{{\cellcolor{shadecolor}}}

\newcommand{\notfound}{\cellcolor{black}\color{white}}

% Comandos específicos para o Tikz
\newcommand{\bloco}[5]{
    % row, t, size, color, label
    \filldraw[
        fill=#4!30!white,
        draw=#4,
    ]
        (#2+1,#1) rectangle (#2+#3+1,#1+1)
        node at (#2 + 1.5,#1 + 0.5) {#5}
    ;
}

\newcommand{\blocoR}[2]{
    % t, label
    \bloco{6}{#1}{1}{red}{#2}
}
\newcommand{\blocoT}[2]{
    % t, label
    \bloco{6}{#1}{1}{gray}{#2}
}
\newcommand{\blocoE}[3]{
    % row, t, size
    \bloco{6-#1}{#2}{#3}{blue}{$E_#1$}
}

\chapter{Conclusões e perspectivas~\label{chp:conclusions}}

A estratégia de utilizar a API JSON não depende de reconhecimento de padrões
textuais e sintáticos para obtenção de dados. Apesar de esta estratégia se
demonstrar mais fácil e eficiente de se implementar e utilizar, ambas foram
mantidas na ferramenta para a extração de dados em TJs que não possuam uma API
JSON como a do TJ-RJ.

\todo{Conclusões da parte dos TJs:}
\begin{todolist}
    \item Comentar o impacto da informatização de um sistema para além do
          público-alvo direto do sistema (ex: quem se beneficia não é somente
          um juíz, procurador, etc., mas também jornalistas que precisam de
          estatísticas relacionadas aos processos).
    \item Comentar sobre o impacto de decisões de formatos de distribuição e
          APIs no processo de informatização.
\end{todolist}

\todo{Conclusões da parte das bibliotecas:}
\begin{todolist}
    \item Comentar da dificuldade de testagem com Scrapy (por conta do Twisted).
\end{todolist}

\todo{%
    Em algum lugar, descrever que na prática a cache passa a ser a finalidade
    da extração e a exportação fica como um mero detalhe.
}

\todo{Conclusões dos resultados experimentais:}
\begin{todolist}
    \item Tempo de IO realmente era o maior gargalo inicial para o tempo de consulta;
    \item ...
\end{todolist}

\todo{Perspectivas:}
\begin{todolist}
    \item Expandir para outros TJs.
    \item Tratar processos em múltiplas instâncias.
\end{todolist}

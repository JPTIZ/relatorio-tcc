\begin{resumo}
    Devido ao aumento da disponibilidade de informações decorrente das
    tecnologias modernas de comunicação, o jornalismo de dados surge com o
    objetivo de avaliar essas informações, reorganizá-las e humanizá-las ao
    público. Para dados processuais, porém, há uma carência de acesso
    organizado a eles por parte dos portais dos Tribunais de Justiça do Brasil.
    Utilizando técnicas de raspagem de dados, neste trabalho foi construída uma
    ferramenta, nomeada \tjscraper, que extrai os dados de todos os processos
    do Tribunal de Justiça do Rio de Janeiro. Para otimização dessa extração,
    foram utilizadas técnicas de programação assíncrona via corrotinas,
    \textit{caching} dos processos além do aproveitamento do sistema de
    numeração unificado do Conselho Nacional de Justiça para redução
    significativa do trabalho necessário para descoberta de processos. A
    aplicação de tais técnicas se mostrou capaz de tornar o tempo de extração
    de processos de um ano específico hábil. Ao final, são discutidos alguns
    dos impactos das decisões de projeto e perspectivas de futuras otimizações
    e aspectos para se expandir do projeto.

    \vspace{\onelineskip}
    \noindent
    \textbf{Palavras-chave}: jornalismo investigativo, web, tribunal de
    justiça, jornalismo de dados, consultas processuais.
\end{resumo}

\afterpage{\null\newpage}

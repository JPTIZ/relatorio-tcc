\chapter{Conclusões e perspectivas~\label{chp:conclusions}}

Há pelo menos quatro pontos de vista que se permitem ser analisados neste
trabalho: do propósito da ferramenta \tjscraper, dos resultados obtidos pelos
experimentos, das decisões de projeto de implementação dela, e das escolhas de
terceiros na modelagem e apresentação de dados da fonte de extração.

No que se refere ao propósito da ferramenta, por ser capaz de obter os dados
todos os processos de um Tribunal de Justiça, a partir do momento que se tem
esses dados de alguma forma local independente do próprio portal do TJ há já um
auxílio para jornalistas investigativos uma vez esses dados, convertidos para
um formato de familiaridade do investigador, podem ser reorganizados da maneira
que for mais conveniente para sua análise. Há, nisso, um detalhe curioso de que
ainda que o propósito original em essência da ferramenta seja a exportação dos
dados em um formato como uma planilha XLSX, o processo de implementação troca a
finalidade para ser alimentar e gerenciar corretamente da \textit{cache} --- na
prática uma replicação local de um banco de dados ---, sendo a exportação é um
mero detalhe como um passo adicional desse processo.

Pelos resultados experimentais ficou confirmada a hipótese de que o gargalo do
tempo de extração dos dados de processos estaria no tempo de IO (especialmente
o de Rede). Atacando esse gargalo, as técnicas de aceleração de consulta
aplicadas neste trabalho cumpriram seu objetivo de reduzir significativamente o
tempo de consulta, chegando a --- considerando apenas IO de Rede --- cerca de
4ms por processo. Ainda que não estejam dentro dos 0.78s mencionados
no~\Cref{chp:construção-da-ferramenta}, se for considerado que processos do
TJ-RJ costumam totalizar menos de 2 milhões de
processos~\cite{conjur:tj-novos-processos,amaerj:tj-novos-processos} e
considerando apenas o tempo de IO de Rede é possível extrair dados de todos os
processos de um ano do TJ-RJ em menos de 30 dias. Uma observação pertinente é
que, não se utilizando de \textit{threads} para a implementação das estratégias
(e sim Corrotinas), a complexidade de implementação foi mínima visto que não
foram necessários mecanismos de sincronização ou controle de condições de
corrida. Dito isso, ainda há espaços para mais otimizações, como o
reordenamento das unidades de origem de forma dinâmica ao longo da extração
pelas que possuirem mais processos registrados (explorando uma potencial
distribuição estatística). Com a presença da \textit{cache}, também, é possível
que o trabalho de extração seja feito de forma distribuida e posteriormente
unidos os registros de \textit{caches} de diferentes máquinas em uma única.

Do ponto de vista das decisões de projeto de implementação da \tjscraper, a
estratégia de utilizar a API JSON, por não depender de reconhecimento de padrões
textuais e sintáticos para obtenção de dados, se demonstrou mais fácil e
eficiente de se implementar e utilizar. Ainda assim, a implementação da
extração HTML servindo como prova de conceito e podendo ser útil para TJs que
não possuam uma API JSON --- contando apenas com a exibição de páginas HTML ---
foi mantida na base do código-fonte.

Na questão de ferramentas, vale ressaltar que foram necessários meios não tão
convencionais~\cite{tjscraper:commit-scrapy-run-spider} de se contornar
problemas de implementação de Scrapy para a elaboração de testes unitários: por
ser uma biblioteca mais antiga que o mecanismo de \texttt{async/await} de
Python 3.5+, a Scrapy se baseou na Twisted, uma biblioteca cuja implementação
se utiliza de estado global e mecanismos de controle que impedem a execução de
uma ou mais \textit{spiders} mais de uma vez em um mesmo processo, dificultando
a implementação de testes unitários (já que cada teste executaria uma
\textit{spider} novamente). Com a implementação da extração via API JSON
utilizando o mecanismo de \texttt{async/await}, porém, é possível reescrever a
extração de processos via HTML utilizando uma biblioteca substituta para o uso
de expressões XPath em cima desse mecanismo.

Além das otimizações possíveis, outros trabalhos que se desenrolam a partir
deste incluem o suporte a outros TJs que não o TJ-RJ, o que demanda a
investigação de quais possuem um processo vigente de informatização, e o
tratamento de processos que possuam múltiplas instâncias.

Por fim, do ponto de vista das escolhas de terceiros, observa-se que as
decisões de um portal de dados e transparência, como é o caso dos TJs, impactam
além do usuário-alvo original. Primeiramente, este trabalho só foi possível
pois há, ainda que de maneira espalhada (múltiplos subdomínios com
funcionamentos diferentes, por exemplo), algum grau de informatização.
\review{Ele, cumprindo sua função, se torna mais uma ferramenta para que outros
setores de pesquisa e/ou jornalísticos possam analisar e explicar à sociedade
sobre fenômenos que ela vive, e portanto informatizar é também dar
conhecimento.} Da mesma forma, porém, ele surge por consequência de uma
``deficiência'' nos mecanismos de busca dos TJs do ponto de vista de facilitar
que sejam feitas análises estatísticas, visto que um mero parâmetro ``Assunto''
nas consultas processuais ou uma forma de listagem de processos facilitaria já
o trabalho de quem faça tais análises ou de implementação de uma ferramenta
semelhante à \tjscraper. Observando as decisões arquiteturais, algumas afetam
de maneira potencialmente não esperada, a exemplo da existência de uma API JSON
que, apesar de criada provavelmente para melhor implementar um sistema de SPA
(\textit{Single-Page-Application}) no subdomínio ww3 do TJ-RJ, foi útil para se
reduzir o esforço necessário para extração dos campos dos dados dos processos.
A falta de documentação pública dessa API, porém, frequentemente se mostrou uma
barreira durante a implementação, uma vez que novos formatos de resposta do
servidor do TJ-RJ eram descobertos para processos específicos, ou mesmo com
relação a não se saber se há formas de simplificação da implementação da
ferramenta devido ao não conhecimento de parâmetros ou rotas que podem existir
na API mas que não são conhecidos.

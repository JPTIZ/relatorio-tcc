\chapter{Trabalhos Correlatos~\label{chp:related-work}}

De outros extratores de Tribunais de Justiça, a~\cite{plexi-api} é uma API
JSON\footnote{``API JSON'', neste trabalho, se refere a serviços Web que
utilizam elementos das requisições HTTP para dar acesso a uma API que responde
com objetos JSON.} que fornece consultas processuais simples a diversos portais
públicos (TJ-RJ incluso), sem focar em jornalismo de dados. As rotas que esta
API possui para acesso a dados do portal do TJ-RJ incluem apenas consultas com
os mesmos parâmetros/campos de busca dele, tendo os acessos feitos ao
subdomínio http://www4.tjrj.jus.br/ (o mesmo apresentado
na~\Cref{sec:extração-html}). Bastante semelhante a esta é a~\cite{intima.ai},
também fornecendo dados processuais apenas sabendo informações prévias sobre
eles (neste, necessariamente o número do processo), com diferenças na forma de
uso. A~\cite{codilo-api}, também uma API, possui uma intersecção maior com os
recursos deste trabalho: alimenta uma base de dados local, fornece
\textit{softwares} extratores de dados jurídicos, e permite a consulta de
processos. Não é claro em sua apresentação sobre detalhes a respeito de quais
dados os extratores são responsáveis por capturar, dificultando um levantamento
de seus limites. O que se sabe, porém, é que as fontes de extração são portais
diferentes dos utilizados no \tjscraper, como o PJe (Processo Judicial
Eletrônico) para processos trabalhistas e o Projudi para os estaduais. Essas
três ferramentas não são abertas nem de \textbf{Software-Livre}\footnote{Um
\textbf{Software-Livre} é aquele atende a quatro liberdades
essenciais~\cite{def:free-software}: a de se executá-lo como desejar, de
estudá-lo e modificá-lo, de redistribuí-lo, e de redistribuir cópias de edições
modificadas dele.}. São, inclusive, pagas (a~\cite{intima.ai} através da compra
de créditos que são gastos ao longo das consultas). A
~\Cref{tab:comparação-ferramentas-relacionadas} resume as diferenças entre as
ferramentas descritas.


\begin{table}[htb]
    \tiny
    \centering
    \begin{tabular}{p{0.2\textwidth}p{0.16\textwidth}p{0.16\textwidth}p{0.16\textwidth}p{0.16\textwidth}}
        \toprule
        \textbf{Característica} & \textbf{\cite{plexi-api}} & \textbf{\cite{intima.ai}} & \textbf{\cite{codilo-api}} & \textbf{\tjscraper} \\
        \midrule
        Tipo & API Web & API Web & API Web & Biblioteca/API Web \\
        Licença & Não especificada & Não especificada & Não especificada & GPL-v3 \\
        Parâmetros de busca & Mesmos do TJ-RJ & Número do Processo & Não especificado & Numeração Unificada, Assunto, Ano \\
        Campos dos processos disponibilizados & Todos & Todos & Não especificado & Todos \\
        Restrição de acesso a dados & Token de acesso (pago) & Crédito (pago) por busca & Negociação com a empresa & Nenhuma \\
        Banco de Dados local & Não & Não & Integração com BD pré-existente do cliente & Sim (próprio da ferramenta) \\
        \toprule
        \textbf{Fonte (TJ-RJ)} & & & \\
        \midrule
        Projudi &   &   & \checkmark \\
        PJe & \checkmark & \checkmark & \checkmark \\
        Consulta Processual & \checkmark & Não especificado & \checkmark & \checkmark \\
        \bottomrule
    \end{tabular}
    \caption{%
        Comparação de características de outras ferramentas com objetivos
        semelhantes à TJScraper.
    }
    \label{tab:comparação-ferramentas-relacionadas}
\end{table}

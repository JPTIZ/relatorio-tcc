\section{Método de Desenvolvimento}

O levantamento e avaliação de requisitos deste trabalho foi levantado com base
no contato direto com dois profissionais da área de jornalismo investigativo
através de reuniões ao longo de seu desenvolvimento. Pelo fato do Tribunal de
Justiça do Estado do Rio de Janeiro (TJ-RJ) ser o mais informatizado no
território brasileiro, ele é utilizado como ponto de partida da produção da
ferramenta \tjscraper~focando apenas nos seus processos em primeira instância
para simplificação.

O funcionamento das páginas, rotas e APIs disponibilizadas pelo portal do TJ-RJ
são descobertos via tentativa e erro, em especial para as APIs pela falta de
documentação disponível. O portal do TJ-RJ possui uma restrição em que apenas
no período das 23h00 às 7h30, período o qual neste trabalho será referido como
``horário noturno'', está disponível para ferramentas de raspagem de dados para
acessá-lo livremente. Fora dessa faixa de horário, o acesso é bloqueado por
meio de \textit{captchas} (pequenos ``quebra-cabeças'' que devem ser resolvidos
para se obter acesso a uma página e os quais são inviáveis para softwares).
Para contornar o horário noturno e permitir a produção da ferramenta em outros
horários, foram criados testes unitários ao longo do desenvolvimento a partir
de amostras de respostas de diferentes rotas do TJ-RJ.

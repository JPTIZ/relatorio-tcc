\section{Objetivo e Escopo}

Visando resolver a demanda por ferramentas de consulta processual mais
abrangentes, neste trabalho se propôs a construção de uma ferramenta de
raspagem de dados especializada em tribunais de justiça do Brasil de forma que
a adição de suporte a um novo tribunal de justiça não acarrete em reescrever
significativamente a ferramenta. A ferramenta, nomeada \textbf{\tjscraper},
para o escopo deste trabalho se limita a apenas processos em primeira
instância do Tribunal de Justiça do Estado do Rio de Janeiro (TJ-RJ).

Este trabalho também propõe, para implementação da ferramenta, estratégias para
aceleração das consultas que ataquem os seguintes problemas: o acúmulo do tempo
de espera entre o envio de requisições e o recebimento de suas respostas, e o
retrabalho causado por consultas subsequentes cujos processos que extrairiam se
interseccionem com processos que já foram previamente extraídos por uma
consulta anterior. Para tal, deve-se ter também a construção da especificação
de uma base de dados que considere a possibilidade de exportação para outros
formatos.

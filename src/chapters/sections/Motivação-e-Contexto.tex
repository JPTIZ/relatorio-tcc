\section{Motivação e Contexto~\label{sec:Motivação-e-Contexto}}

\todo{Revisar o problema:}
\begin{todolist}
    \item Contextualizar sobre o papel do Jornalismo Investigativo;
    \item Comentar ligeiramente sobre as dificuldades de jornalistas na
          obtenção de dados. Para refinar o argumento: citar um exemplo de
          notícia que faça correlações de dados que só faça sentido/seja viável
          com esses dados resumidos e tabelados;
    \item Explicitar com o caso de Tribunais Regionais.
\end{todolist}

No jornalismo, há notícias de diferentes categorias de notícias que podem ser
apresentadas, dentre elas as de caráter
\todofootnote{estatístico/comparativo}{Revisar nomenclatura. Adicionar uma
referência}. Tal categoria exigem do jornalista acesso a conjuntos de dados que
por vezes se encontra disponível em formas difíceis de se agregar para elaborar
estatísticas, seja por estarem dispersos ao longo de diferentes fontes de
informação .??? :D como é o caso dos Tribunais de Justiça em que

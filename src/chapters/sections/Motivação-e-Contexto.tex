\section{Motivação e Contexto~\label{sec:Motivação-e-Contexto}}

% \todo{Comunicação social}
% \todo{Mencionar sobre o pauteiro}
% \todo{O que o JD faz é \textit{humanizar} esses dados, ou seja, pegar os dados e contar uma história.}
% \todo{no Data-Driven Journalism você já tem as 6 perguntas do jornalismo respondidas, mas não tem o ``e daí?''.}

O avanço das tecnologias de comunicação, gerando uma abundância maior de
informações bastante notória principalmente com a popularização da internet,
traz à área de jornalismo mudanças em seu modo de produção criando o que
popularmente se chama ``jornalismo de dados''. A direção convencional da
elaboração de matérias jornalísticas é que haja primeiro a presença do fenômeno
e, posteriormente, se buscam dados e informações que complementem a descrição
dele. No jornalismo de dados, a direção é a oposta: primeiro se tem os dados e,
a partir deles, se conclui sobre a existência de um fenômeno que então é
noticiado. % FIXME: Insira aqui uma reportagem.
É exigido do jornalista, nesse caso, que ele tenha acesso a esses
dados que por vezes se encontra disponível em formas difíceis de se agregar
para elaborar estatísticas, seja por estarem dispersos ao longo de diferentes
fontes de informação ou simplesmente não se é viável analisá-los somente pela
forma como estão disponibilizados, como é o caso dos sites dos Tribunais
Regionais de Justiça como o do Rio de Janeiro~\cite{tjrj} em que não se há uma
maneira nem de realizar consultas processuais a partir de parâmetros mais
abertos, como por assunto, nem de se obter alguma forma de resumo de todos os
processos registrados.

Para casos como esse em que um portal de informações não as disponibiliza de
forma conveniente, uma forma de contornar essa deficiência é através de
raspagem de dados, ou seja, o uso de ferramentas de software que façam a
varredura de páginas da internet buscando e reorganizando dados de interesse de
forma que possam ser posteriormente processados. O uso dessas ferramentas,
chamadas de ``extratores'', para tais finalidades dentro do jornalismo de
dados, aliado a outras práticas, caracteriza o ``jornalismo investigativo''.
Para os Tribunais de Justiça (TJs), porém, não há --- ao menos acesso fácil e
público --- esse tipo de ferramenta, dificultando o trabalho de jornalistas
investigativos na obtenção e análise de processos.

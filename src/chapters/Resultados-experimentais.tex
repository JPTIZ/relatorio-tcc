\chapter{Resultados experimentais~\label{chp:Resultados-Experimentais}}

\begin{figure}[htb]
    \caption{%
        Relação entre o tamanho do passo de requisições e processos adquiridos
        por \todo{segundo\protect\footnotemark{}}.
    }
    \label{gra:tempos-tamanhos-de-passo-async}
\end{figure}

\footnotetext{\todo{Verificar se por minuto não fica mais adequado (se o tempo de
processamento em um segundo deixar muitos processos pendentes, no caso)}}

\todo{%
    Comparar um cenário base (sem aplicar as estratégias) com:
}
\begin{todolist}
    \item Apenas uma das estratégias aplicadas (fazer para cada estratégia);
    \item Com todas as estratégias aplicadas.
\end{todolist}
\todo{%
    A hipótese lançada é de que o maior tempo desprendido é com IO (e não
    processamento das requisições).
    A conclusão deverá confirmar essa hipótese demonstrando que com a redução
    (através de caching e através de requisições assíncronas) do gasto com IO o
    tempo de consulta se reduz de forma diretamente proporcional.
}


\begin{figure}[htb]
    \centering
    \begin{tikzpicture}
        \begin{axis}[
            ybar,
            xmin=-20,
            xmax=130,
            xtick={10, 100, 1000},
            ylabel={Tempo (s)},
            xlabel={Tamanho do bloco},
            legend cell align=left,
            legend style={
                legend pos=outer north east,
                cells={align=left},
            },
        ]
            \pgfplotstableread{io_stats-async-sync.csv}{\table}
            \pgfplotstablegetcolsof{\table}
            \pgfmathtruncatemacro\numberofcols{\pgfplotsretval-1}
            \pgfplotsinvokeforeach{1,...,\numberofcols}{
                \pgfplotstablegetcolumnnamebyindex{#1}\of{\table}\to{\colname}
                \addplot table [y index=#1] {\table};
                \addlegendentryexpanded{\colname}
            }
        \end{axis}
    \end{tikzpicture}
    \caption{%
      Comparativo de tempo desprendido com IO de Rede e processamento em CPU
      entre o cenário com e sem a aplicação de programação assíncrona via IO
      não-bloqueante (Async e Sync, respectivamente).
    }
    \label{gra:tempos-async-vs-sync}
\end{figure}

\todo{%
  Nos resultados experimentais, ver se o tempo médio de obtenção de processos
  ficou nessa linha dos 0.16ms lá.
}

\begin{todolist}
    \item Adicionar gráfico comparando o uso da cache para a mesma consulta com
          quando ela é feita com e sem caching.
\end{todolist}

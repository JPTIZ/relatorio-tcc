\chapter{Análises Experimentais~\label{chp:Resultados-Experimentais}}

Nesta seção estão descritos os resultados obtidos a partir de experimentos
visando como objetivo principal averiguar a capacidade da ferramenta de obter e
exportar dados de processos de um TJ em tempo viável. Para isso, adotou-se como
critérios de análise o \textbf{tempo total da consulta} de
um intervalo de processos, separados entre tempo de CPU, IO de Rede e IO de
Armazenamento, e o \textbf{tempo médio} para se descobrir um processo em um
intervalo.

\section{Configuração Experimental~\label{sec:Configuração-Experimental}}

Os experimentos foram realizados em uma estação de trabalho com um Intel(R)
Core(TM) i5-7300HQ CPU @ 2.50GHz, 16GB RAM DDR4 2133MHz e interface de rede
1000Mbps em uma conexão residencial.

Os dados de temporização foram obtidos através da média dos tempos de 3
execuções do perfilamento de uma chamada completa do processo de extração via
API JSON do TJ-RJ de um intervalo de processos variando os parâmetros com todas
as combinações dos valores especificados
na~\Cref{tbl:parâmetros-de-perfilamento}, com o intervalo inicial (15712)
escolhido arbitrariamente. Nas combinações com tamanho do lote em 1, o cenário
é classificado como Síncrono (Sync), e os demais como Assíncrono (Async).

\begin{table}[htb]
  \centering
  \begin{tabular}{ll}
    \toprule
    Parâmetro & Valores \\
    \midrule
    Número inicial do intervalo & 15712 \\
    Tamanho do intervalo do campo ``NNNNNNN'' (nº de processos) & 10, 50, 100, 1000 \\
    Tamanho do lote & 1, 10, 100, 500, 1000 \\
    \bottomrule
  \end{tabular}
  \caption{%
      Valores utilizados para os parâmetros de extração de dados de processos.
  }
  \label{tbl:parâmetros-de-perfilamento}
\end{table}

\section{Resultados Experimentais}

A~\Cref{gra:tempos-async-vs-sync} mostra um comparativo do tempo decorrido para
uma implementação síncrona e uma assíncrona nos diferentes cenários de
configuração, com a~\Cref{tbl:tempos-async-vs-sync} expondo o tempo por
processo do mesmo comparativo, obtido dividindo-se o tempo total naquela
configuração pelo número de processos. Percebe-se por esses dados que passando
do cenário síncrono para o assíncrono há uma redução significativa no tempo de
IO de Rede (de 66\% a 98\%) mantendo um tempo de CPU bastante próximo,
demonstrando a eficiência do uso de requisições assíncronas.

É possível perceber que o uso de requisições assíncronas traz uma maior redução
no tempo médio com uma quantidade maior de processos, uma vez que um conjunto
de poucos processos tira pouco proveito da assincronia não substituindo o tempo
de ociosidade com o tempo de se disparar mais requisições, enquanto um número
maior é capaz de preencher esse tempo.

\begin{figure}[htb]
    \centering
    \begin{tikzpicture}
        \pgfplotstableread{io_stats-async-sync.csv}{\table}
        \pgfplotstablegetcolsof{\table}
        \pgfmathtruncatemacro\numberofcols{\pgfplotsretval-1}
        \pgfplotstablegetcolumnnamebyindex{0}\of{\table}\to{\colprincipal}
        \begin{axis}[
            ybar,
            enlarge x limits=0.15,
            symbolic x coords={
              10, 50, 100, 1000,
            },
            xtick=data,
            xticklabels from table={\table}{\colprincipal},
            ylabel={Tempo (s)},
            xlabel={Nº de processos},
            legend cell align=left,
            legend style={
                legend pos=outer north east,
                cells={align=left},
            },
            width=0.7\textwidth,
            height=14em,
            clip=false,
        ]
            \pgfplotsinvokeforeach{1,...,\numberofcols}{
                \pgfplotstablegetcolumnnamebyindex{#1}\of{\table}\to{\colname}
                \addplot table [y index=#1] {\table};
                \addlegendentryexpanded{\colname}
            }
        \end{axis}
    \end{tikzpicture}
    \caption{%
      Comparativo de tempo despendido com IO de Rede e processamento em CPU
      entre o cenário sem a aplicação de programação assíncrona via IO
      não-bloqueante (Sync) e com a aplicação (Async).
    }
    \label{gra:tempos-async-vs-sync}
\end{figure}

\begin{table}[htb]
  \centering
  \begin{tabular}{lllll}
    \toprule
     & \multicolumn{2}{c}{Sync} & \multicolumn{2}{c}{Async} \\
    Nº de processos & IO de Rede & CPU & IO de Rede & CPU \\
    \midrule
    10 & 0.1850 & 0.0039 & 0.0158 & 0.0020 \\
    50 & 0.2386 & 0.0104 & 0.0022 & 0.0026 \\
    100 & 0.2021 & 0.0106 & 0.0012 & 0.0030 \\
    1000 & 0.2272 & 0.0061 & 0.0008 & 0.0032 \\
    \bottomrule
  \end{tabular}
  \caption{%
    Tempo médio de busca por processo comparando os cenários Síncrono e
    Assíncrono.
  }
  \label{tbl:tempos-async-vs-sync}
\end{table}

Quanto à separação das requisições em lotes, as medições de temporização estão
expostas na~\Cref{gra:tempos-tamanhos-de-passo-async}. Os resultados para um
intervalo 10000 processos estão separados
na~\Cref{gra:tempos-tamanhos-de-passo-async-10000-processos} para melhor
visualização. De maneira geral, tamanhos de lote maiores ou iguais que 100
tiveram resultados similares para quase todos os tamanhos de intervalo de
processos, tendo como excepcionalidade o intervalo de 10 processos com tamanho
do lote em 500. Essa excepcionalidade, porém serve apenas para uma observação
comparativa, visto que TJs possuem na ordem de centenas de milhares de
processos por ano.

Para analisar essas medidas é preciso levar em conta que, para evitar sanções e
sem acesso a um \texttt{robots.txt}, não é possível ter sempre o tamanho do
lote igual ao do intervalo de processos. Partindo dessa premissa, as medidas
expõem que, ao menos para um intervalo de até 1000 processos com tamanhos de
lote maiores ou iguais a 100, não há ganhos significativos em se manter o
tamanho de lote tão próximo do tamanho do intervalo, porém para os casos de
intervalos de tamanho 1000 e 10000 os lotes de tamanho 500 se demonstraram
suficientes e com os melhores resultados.

\begin{figure}[htb]
    \centering
    \begin{tikzpicture}
        \pgfplotstableread{io_stats-batch-size.csv}{\table}
        \pgfplotstablegetcolsof{\table}
        \pgfmathtruncatemacro\numberofcols{\pgfplotsretval-1}
        \pgfplotstablegetcolumnnamebyindex{0}\of{\table}\to{\colprincipal}
        \begin{axis}[
            ybar,
            enlarge x limits=0.15,
            symbolic x coords={
                10, 100, 500, 1000
            },
            ylabel={Tempo (s)},
            xlabel={Tamanho do lote},
            legend cell align=left,
            legend style={
                legend pos=outer north east,
                cells={align=left},
            },
            xtick=data,
            xticklabels from table={\table}{\colprincipal},
            width=0.7\textwidth,
            clip=false,
        ]
            \pgfplotsinvokeforeach{1,...,\numberofcols}{
                \pgfplotstablegetcolumnnamebyindex{#1}\of{\table}\to{\colname}
                \addplot table [y index=#1] {\table};
                \addlegendentryexpanded{\colname}
            }
        \end{axis}
    \end{tikzpicture}
    \caption{%
        Comparação do tempo de obtenção de todos os processos de diferentes
        intervalos conforme tamanho do lote.
    }
    \label{gra:tempos-tamanhos-de-passo-async}
\end{figure}

Analisando-se o tempo de espera por IO de Rede
na~\Cref{gra:tempos-tamanhos-de-passo-async-10000-processos}, dado em uma média
aritmética por processo, percebe-se também que um tamanho de lote muito grande
(nesse caso, 1000) acarreta em um aumento no tempo de espera, o que deve
explicar o fato de que o tempo total de consulta passa a aumentar com relação
aos lotes de tamanho 500.

\begin{figure}[htb]
    \centering
    \begin{tikzpicture}
        \pgfplotstableread{io_stats-batch-size-10000.csv}{\table}
        \pgfplotstablegetcolsof{\table}
        \pgfmathtruncatemacro\numberofcols{\pgfplotsretval-1}
        \pgfplotstablegetcolumnnamebyindex{0}\of{\table}\to{\colprincipal}
        \begin{axis}[
            ybar,
            enlarge x limits=0.15,
            symbolic x coords={
                100, 500, 1000
            },
            bar shift=-0.5em,
            ylabel={Tempo total (s)},
            xlabel={Tamanho do lote},
            axis y line*=left,
            xtick=data,
            ymin=0,
            xticklabels from table={\table}{\colprincipal},
            width=0.7\textwidth,
            height=14em,
            clip=false,
            %legend style={at={(1,1)},anchor=north west},
            legend style={at={(0,1.1)},anchor=south west},
        ]
            \pgfplotsinvokeforeach{1,...,\numberofcols}{
                \pgfplotstablegetcolumnnamebyindex{#1}\of{\table}\to{\colname}
                \addplot table [y index=#1] {\table};
                \legend{Tempo total};
            }
        \end{axis}
        \begin{axis}[
            ybar,
            enlarge x limits=0.15,
            symbolic x coords={
                100, 500, 1000
            },
            bar shift=0.5em,
            axis y line*=right,
            ylabel={Tempo (IO de Rede) (s)},
            xlabel={Tamanho do lote},
            xtick=data,
            ymin=0,
            xticklabels from table={\table}{\colprincipal},
            width=0.7\textwidth,
            height=14em,
            clip=false,
            legend style={at={(1,1.1)},anchor=south east},
        ]
          \addplot [
            red,
            fill=red!35,
          ] coordinates {
            (100, 1.27)
            (500, 1.02)
            (1000, 4.47)
          };
          \legend{Tempo de espera por IO de Rede};
        \end{axis}
    \end{tikzpicture}
    \caption{%
        Comparação do tempo de obtenção de todos os processos de um intervalo
        de tamanho 10000 conforme tamanho do lote.
    }
    \label{gra:tempos-tamanhos-de-passo-async-10000-processos}
\end{figure}

Para medir a eficiência da \textit{cache}, foram separados três intervalos de
valores para ``NNNNNNN'' com intersecção, $I_1 = [15712, 16712]$, $I_2 =
[16212, 17212]$ e $I_3 = I_1 \cup I_2$, e 3 cenários
diferentes~\Cref{tab:descrição-cenários-cache} com os mesmos parâmetros de
consulta (com exceção do intervalo de processos). As medidas de eficiência,
dadas em termos do tempo total de execução e do tempo gasto com IO de Rede
(ambos em segundos) estão expostas na~\Cref{gra:tempos-cache}. A execução de
cada cenário se dava com uma \textit{cache} inicialmente vazia.

No cenário com intersecção completa (cenário $C$), como esperado, na consulta
$C_2$ o tempo de IO de Rede é zerado visto que todos os processos que $C_2$
baixaria já foram previamente salvos na \textit{cache} por $C_1$ e portanto
sendo carregados dela em vez da rede. É interessante notar que, com isso, o
tempo total da consulta também se reduziu drasticamente, demonstrando que o uso
da \textit{cache} foi compensatório. No cenário com intersecção parcial
(cenário $A$), novamente o tempo total de consulta também se reduziu
significativamente além do tempo de IO de Rede. Comparando as consultas $A_2$ e
$B$ (que operam sob o mesmo intervalo, $I_2$), reforçam-se as conclusões
analisadas em $C$: para um mesmo intervalo, a \textit{cache} ainda que parcial
foi capaz de reduzir ambos o tempo de IO de rede e o tempo total da consulta,
sendo novamente compensatória.

\begin{table}[htb]
  \begin{tabular}{cp{0.4\textwidth}p{0.4\textwidth}}
    \toprule
    Cenário & Objetivo & Características \\
    \midrule
    A
    &
    Medir a eficiência da cache em reduzir parte do retrabalho.
    &
    Separado em duas consultas feitas em sequência: $A_1$ (intervalo $I_1$) e
    $A_2$ (intervalo $I_2$). A \textit{cache} preenchida por $A_1$ é
    reutilizada para $A_2$.
    \\
    B & Medir o tempo de $A_2$ sem \textit{cache}. & Uma única consulta no intervalo $I_2$.
    \\
    C
    &
    Averiguar se há compensação em utilizar a \textit{cache} para reduzir o
    tempo de busca em caso de retrabalho.
    &
    Separado em duas consultas, $C_1$ e $C_2$, no intervalo $I_3$
    reaproveitando a \textit{cache} preenchida por $C_1$ em $C_2$.
    \\
    \bottomrule
  \end{tabular}
  \caption{Descrição dos cenários utilizados para medição da eficiência da \textit{cache}.}
  \label{tab:descrição-cenários-cache}
\end{table}

\begin{figure}[htb]
    \centering
    \begin{tikzpicture}
        \pgfplotstableread{io_stats-cache-effect.csv}{\table}
        \pgfplotstablegetcolsof{\table}
        \pgfmathtruncatemacro\numberofcols{\pgfplotsretval-1}
        \pgfplotstablegetcolumnnamebyindex{0}\of{\table}\to{\colprincipal}
        \begin{axis}[
            ybar,
            bar shift=-0.5em,
            enlarge x limits=0.15,
            symbolic x coords={
                A1, A2, B, C1, C2,
            },
            axis y line*=left,
            ylabel={Tempo total (s)},
            xlabel={Consulta},
            legend cell align=left,
            legend style={
              at={(0,1.1)},
              anchor=south west,
            },
            xtick=data,
            xticklabels from table={\table}{\colprincipal},
            ymin=0,
            width=0.7\textwidth,
            clip=false,
        ]
            \pgfplotstablegetcolumnnamebyindex{1}\of{\table}\to{\colname}
            \addplot table [y index=1] {\table};
            \addlegendentryexpanded{\colname}
        \end{axis}
        \begin{axis}[
            ybar,
            bar shift=0.5em,
            enlarge x limits=0.15,
            symbolic x coords={
                A1, A2, B, C1, C2,
            },
            axis y line*=right,
            ylabel={Tempo de espera por IO de Rede (s)},
            legend cell align=left,
            legend style={
              at={(1,1.1)},
              anchor=south east,
            },
            xtick=data,
            xticklabels from table={\table}{\colprincipal},
            ymin=0,
            width=0.7\textwidth,
            clip=false,
        ]
            \pgfplotstablegetcolumnnamebyindex{2}\of{\table}\to{\colname}
            \addplot [red, fill=red!35] table [y index=2] {\table};
            \addlegendentryexpanded{\colname}
        \end{axis}
    \end{tikzpicture}
    \caption{%
      Comparação do tempo de obtenção de todos os processos em 5 consultas
      específicas a fim de demonstrar a eficiência do uso de \textit{cache}.
    }
    \label{gra:tempos-cache}
\end{figure}

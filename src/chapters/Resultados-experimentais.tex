\chapter{Resultados experimentais~\label{chp:Resultados-Experimentais}}

\begin{figure}[htb]
    \caption{%
        Relação entre o tamanho do passo de requisições e processos adquiridos
        por \todo{segundo\protect\footnotemark{}}.
    }
    \label{gra:tempos-tamanhos-de-passo-async}
\end{figure}

\footnotetext{\todo{Verificar se por minuto não fica mais adequado (se o tempo de
processamento em um segundo deixar muitos processos pendentes, no caso)}}

\todo{%
    Comparar um cenário base (sem aplicar as estratégias) com:
}
\begin{todolist}
    \item Apenas uma das estratégias aplicadas (fazer para cada estratégia);
    \item Com todas as estratégias aplicadas.
\end{todolist}
\todo{%
    A hipótese lançada é de que o maior tempo desprendido é com IO (e não
    processamento das requisições).
    A conclusão deverá confirmar essa hipótese demonstrando que com a redução
    (através de caching e através de requisições assíncronas) do gasto com IO o
    tempo de consulta se reduz de forma diretamente proporcional.
}

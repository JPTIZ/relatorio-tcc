\documentclass[]{ufsc-thesis}

%-------------------------------------------------------------------------------
\usepackage[tcc]{ufsc-thesis-a4}

%----------------------------------------------------------------------
% Pacotes usados especificamente neste documento
\usepackage{adjustbox}
\usepackage{afterpage}
\usepackage{amsmath}
\usepackage{amssymb}
\usepackage{bm}
\usepackage{booktabs}
\usepackage[brazilian]{cleveref}
\usepackage{colortbl}
\usepackage{color}
\usepackage{enumitem}
\usepackage{footnote}
\usepackage{framed}
\usepackage{graphicx}
\usepackage{hhline}
\usepackage{latexsym}
\usepackage{listings}
\usepackage{makecell}
\usepackage{mathtools}
\usepackage{marginnote}
\usepackage{multirow}
\usepackage{pdfpages}
\usepackage{tabularx}
\usepackage[table]{xcolor}
\usepackage[normalem]{ulem}
\usepackage[lined,boxed,ruled,commentsnumbered,portuguese]{algorithm2e}

\usepackage{lmodern}
\usepackage[T1]{fontenc}
% \usepackage{mathptmx}

\makesavenoteenv{tabular}
\makesavenoteenv{table}


\usepackage{tikz}
\usetikzlibrary{decorations.pathreplacing}
\usepackage{standalone}

\newcommand{\specialcell}[2][c]{%
    \begin{tabular}[#1]{@{}c@{}}#2\end{tabular}
}

\DeclarePairedDelimiter{\ceil}{\lceil}{\rceil}

%------------------------------------------------------------------------------
% Configurações do UFSC-Thesis para TCC

%----------------------------------------------------------------------
% Comandos criados pelo usuário
%----------------------------------------------------------------------
\renewcommand{\bf}[1]{\textbf{#1}}
\newcommand{\review}[1]{{\color{green!65!black}{#1}}}
\newcommand{\remove}[1]{{\sout{#1}}}
\newcommand{\newcontent}[1]{{\color{blue}{#1}}}
\newcommand{\tolc}[1]{{\color{red}{#1}}}
\newcommand{\critical}[1]{{\color{red}\textbf{{#1}}}}
\newcommand{\verycritical}[1]{{\color{red}\textbf{\uppercase{{#1}}}}}

\newcommand{\blue}[1]{{\color{blue}{#1}}}
\newcommand{\green}[1]{{\color{green!65!black}{#1}}}

% For to-do listing
\newcommand{\todo}[1]{\textbf{\textit{\color{red}{#1}}}}

\newenvironment{todolist}
{%
    \begin{itemize}
        \color{red}
}{%
    \end{itemize}
}

\newcommand{\todofootnote}[2]{\todo{#1}\footnote{\todo{#2}}}

% Regras de coloração de tabela

\definecolor{shadecolor}{rgb}{0.8,0.8,0.8}
\newcommand\VRule[1][\arrayrulewidth]{\vrule width #1}

\newcommand{\shadecell}{{\cellcolor{shadecolor}}}

\newcommand{\notfound}{\cellcolor{black}\color{white}}

\titulo{Extração de dados processuais dos Tribunais de Justiça para auxílio ao Jornalismo Investigativo (título provisório)}
\autor{João Paulo Taylor Ienczak Zanette}
\data{\today}
\instituicao{Universidade Federal de Santa Catarina}
\local{Florianópolis}
\tipotrabalho{Trabalho de Conclusão de Curso}
\orientador{Profa.\ Dra.\ Carina Friedrich Dorneles}
\programa{Curso de Bacharelado em Ciências da Computação}
\centro{Departamento de Informática e Estatística}

\def\bancaMembroA{\todo{A definir}}
\def\bancaMembroB{\todo{A definir}}

%-------------------------------------------------------------------------------
\preambulo{\imprimirtipotrabalho\ submetido ao \imprimirprograma\ para a
           obtenção do Grau de Bacharel em Ciências da Computação.}
\assuntos{%
    jornalismo investigativo, web, tribunal de justiça, \todo{completar}
}


%*******************************************************************************
\begin{document}

%===============================================================================
\pretextual%
\imprimircapa%

\imprimirfolhaderosto%

\afterpage{\null\newpage}

%===============================================================================
\begin{resumo}
    \todo{Adicionar resumo}

    \vspace{\onelineskip}
    \noindent
    \textbf{Palavras-chave}: jornalismo investigativo, web, tribunal de justiça, \todo{completar}.
\end{resumo}

\afterpage{\null\newpage}


\begin{KeepFromToc}
    \tableofcontents
\end{KeepFromToc}

%===============================================================================
\chapter{Introdução~\label{chp:intro}}

\section{Motivação e Contexto}

\todo{Revisar o problema:}
\begin{todolist}
    \item Contextualizar sobre o papel do Jornalismo Investigativo;
    \item Comentar ligeiramente sobre as dificuldades de jornalistas na
          obtenção de dados. Para refinar o argumento: citar um exemplo de
          notícia que faça correlações de dados que só faça sentido/seja viável
          com esses dados resumidos e tabelados;
    \item Explicitar com o caso de Tribunais Regionais.
\end{todolist}


\section{Objetivo e Escopo}

\todo{%
    Adicionar objetivo e escopo.
}

\input{chapters/sections/Método-de-Pesquisa}

\chapter{Trabalhos Correlatos~\label{chp:related-work}}


\chapter{Descrição das Ferramentas~\label{chp:descrição-das-ferramentas}}

\todo{%
    Esta seção deve ser uma descrição das ferramentas mais relevantes que foram
    \textit{utilizadas} no trabalho, i.e. as dependências (bibliotecas,
    linguagem de programação).
}
\begin{todolist}
    \item Scrapy
    \item SQLite
    \item \ldots
\end{todolist}

\textcolor{blue}{--->> O que estás propondo para descrição de ferramentas (Scrapy e SQLLite) será um pequeno parágrafo de uma Seção chamada "Ferramentas utilizadas" no capítulo de construção da ferramenta. Não vais descrever nenhuma delas com detalhes, pois são ferramentas simplesmente usadas para contruíres teu trabalho. Isso não se detalha porque não são "conceitos". Ferramentas mudam de versão, saem de linha, ficam obsoletas, perdem valor, etc, portanto, não se descrevem em detalhes.}

\chapter{Construção da Ferramenta~\label{chp:construção-da-ferramenta}}

\section{Organização geral}

\begin{todolist}
    \item Explicar a construção da ferramenta como CLI, biblioteca e como
          aplicação Web;
    \item Diagramar módulos;
\end{todolist}

\section{Estratégias de extração}

\subsection{Estratégia inicial: extração HTML}

\begin{todolist}
    \item Descrever a situação inicial das páginas do TJ-RJ;
    \item Ressaltar dificuldades com páginas diferentes, fornecendo links para
          exemplos --- para fins de reprodutibilidade, o link será uma cópia
          para um .html salvo no repositório do código da ferramenta, na pasta
          de ``samples'' (amostras);
    \item Explicar como Scrapy foi utilizado.
\end{todolist}

\subsection{Estratégia alternativa: varredura com API JSON}

\begin{todolist}
    \item Expor sobre a existência de múltiplos sub-domínios do TJ-RJ;
    \item Complementar com a falta de documentação da API do TJ-RJ (forma de
          operação descoberta na base de tentativa e erro);
    \item Explicar rapidamente sobre a filtragem dos campos;
\end{todolist}

\section{Estratégias de aceleração de consulta}

\todo{%
    Para cada estratégia, nomeá-la para referenciar na parte de resultados
    demonstrando os ganhos de aceleração. Nesta seção, não tomar conclusões
    sobre ganhos, apenas enunciar e explicar como as estratégias funcionam e
    por que elas foram decididas dessa forma.
}

\todo{%
    Na seção de resultados, comparar um cenário base (sem aplicar as estratégias) com:
}
\begin{todolist}
    \item Apenas uma das estratégias aplicadas (fazer para cada estratégia);
    \item Com todas as estratégias aplicadas.
\end{todolist}
\todo{%
    A hipótese lançada é de que o maior tempo desprendido é com IO (e não
    processamento das requisições).
    A conclusão deverá confirmar essa hipótese demonstrando que com a redução
    (através de caching e através de requisições assíncronas) do gasto com IO o
    tempo de consulta se reduz de forma diretamente proporcional.
}

\subsection{Requisições assíncronas}

\subsection{\textit{Cache} dos resultados}

\begin{todolist}
    \item Justificar a necessidade de \textit{cache} para os resultados: em
          resumo, requisições, sejam de um mesmo usuário ou de usuários
          diferentes, podem muito bem conter uma intersecção dos processos que
          serão buscados.
\end{todolist}

\chapter{Configuração Experimental~\label{chp:configs}}

\begin{todolist}
    \item Comentar sobre a arquitetura do servidor, incluindo uso de UWSGI em
          um servidor Nginx.
\end{todolist}

\chapter{Resultados experimentais~\label{chp:results}}


\input{chapters/Análise-de-Usabilidade}
\input{chapters/Conclusões-e-perspectivas}

%-------------------------------------------------------------------------------

\bibliography{references}

\end{document}

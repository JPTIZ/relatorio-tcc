\documentclass[
    % -- opções da classe memoir --
    12pt,                 % tamanho da fonte
    %openright,           % capítulos começam em pág ímpar (insere página vazia caso preciso)
    oneside,              % para impressão no anverso. Oposto a twoside
    a4paper,              % tamanho do papel.
    % -- opções da classe abntex2 --
    chapter=TITLE,        % títulos de capítulos convertidos em letras maiúsculas
    section=TITLE,        % títulos de seções convertidos em letras maiúsculas
    %subsection=TITLE,    % títulos de subseções convertidos em letras maiúsculas
    %subsubsection=TITLE, % títulos de subsubseções convertidos em letras maiúsculas
    % -- opções do pacote babel --
    english,              % idioma adicional para hifenização
    %french,              % idioma adicional para hifenização
    %spanish,             % idioma adicional para hifenização
    brazil                % o último idioma é o principal do documento
    ]{abntex2}

%-------------------------------------------------------------------------------
\usepackage[alf]{abntex2cite}

%----------------------------------------------------------------------
% Pacotes usados especificamente neste documento
\usepackage[printonlyused, withpage]{acronym}
\usepackage{adjustbox}
\usepackage{afterpage}
\usepackage{amsmath}
\usepackage{amssymb}
\usepackage{bm}
\usepackage{booktabs}
\usepackage[brazilian]{cleveref}
\usepackage{colortbl}
\usepackage{color}
\usepackage{enumitem}
\usepackage{footnote}
\usepackage{framed}
\usepackage{graphicx}
\usepackage{hhline}
\usepackage{latexsym}
\usepackage{listings}
\usepackage{makecell}
\usepackage{mathtools}
\usepackage{multirow}
\usepackage{pdfpages}
\usepackage{tabularx}
\usepackage[table]{xcolor}
\usepackage[normalem]{ulem}
\usepackage[lined,boxed,ruled,commentsnumbered,portuguese]{algorithm2e}

\makesavenoteenv{tabular}
\makesavenoteenv{table}

\usepackage{fontspec}
\setmainfont{times.ttf}[
    BoldFont = timesbd.ttf,
    ItalicFont = timesi.ttf,
    BoldItalicFont = timesbi.ttf
]

\usepackage{tikz}
\usetikzlibrary{decorations.pathreplacing}
\usepackage{standalone}

\newcommand{\specialcell}[2][c]{%
    \begin{tabular}[#1]{@{}c@{}}#2\end{tabular}
}

\DeclarePairedDelimiter{\ceil}{\lceil}{\rceil}

%------------------------------------------------------------------------------
% Configurações do UFSC-Thesis para TCC
\renewcommand{\imprimircapa}{%
    \begin{capa}%
        \center%
        {\ABNTEXchapterfont\large\MakeUppercase{\imprimirinstituicao}\\
        \ABNTEXchapterfont\large\MakeUppercase{\imprimircentro}}

        \vspace*{1cm}

        {{\normalfont\large\imprimirautor}}

        \vspace*{4cm}
        \begin{center}
            \ABNTEXchapterfont\bfseries\large\MakeUppercase{\imprimirtitulo}
        \end{center}
        \vfill

        \large\imprimirlocal\\
        \large\the\year%
        \vspace*{1cm}
    \end{capa}
}

%----------------------------------------------------------------------
% Comandos criados pelo usuário
%----------------------------------------------------------------------
\renewcommand{\bf}[1]{\textbf{#1}}
\newcommand{\review}[1]{{\color{green!65!black}{#1}}}
\newcommand{\remove}[1]{{\sout{#1}}}
\newcommand{\newcontent}[1]{{\color{blue}{#1}}}
\newcommand{\tolc}[1]{{\color{red}{#1}}}
\newcommand{\todo}[1]{\textbf{\textit{\color{red}{#1}}}}
\newcommand{\critical}[1]{{\color{red}\textbf{{#1}}}}
\newcommand{\verycritical}[1]{{\color{red}\textbf{\uppercase{{#1}}}}}

\newcommand{\blue}[1]{{\color{blue}{#1}}}
\newcommand{\green}[1]{{\color{green!65!black}{#1}}}

\newenvironment{todolist}
{%
    \begin{itemize}
        \color{red}
}{%
    \end{itemize}
}

\definecolor{shadecolor}{rgb}{0.8,0.8,0.8}
\newcommand\VRule[1][\arrayrulewidth]{\vrule width #1}

\newcommand{\shadecell}{{\cellcolor{shadecolor}}}

\usepackage[
    a5paper,
    inner=2.5cm,
    outer=1.5cm,
    top=2.0cm,
    bottom=1.5cm,
    head=0.7cm,
    foot=0.7cm
]{geometry}
\renewcommand{\normalsize}{\small}

\newcommand{\notfound}{\cellcolor{black}\color{white}}

\titulo{A definir}
\autor{João Paulo Taylor Ienczak Zanette}
\data{\today}
\instituicao{Universidade Federal de Santa Catarina}
\local{Florianópolis}
\tipotrabalho{Trabalho de Conclusão de Curso}
\orientador{Prof.\ Dr.\ Luiz Claudio Villar dos Santos}
\programa{Curso de Bacharelado em Ciências da Computação}
\centro{Departamento de Informática e Estatística}

\def\bancaMembroA{\todo{A definir}}
\def\bancaMembroB{\todo{A definir}}

%-------------------------------------------------------------------------------
\preambulo{\imprimirtipotrabalho\ submetido ao \imprimirprograma\ para a
           obtenção do Grau de Bacharel em Ciências da Computação.}
\assuntos{%
    jornalismo investigativo, web, tribunal de justiça
}


\bibliography{references}

%*******************************************************************************
\begin{document}

%===============================================================================
\pretextual%
\imprimircapa%

\imprimirfolhaderosto%

\afterpage{\null\newpage}

%===============================================================================
\begin{resumo}
    \todo{Adicionar resumo}

    \vspace{\onelineskip}
    \noindent
    \textbf{Palavras-chave}: jornalismo investigativo, web, tribunal de justiça, \todo{completar}.
\end{resumo}

\afterpage{\null\newpage}


\begin{KeepFromToc}
    \tableofcontents
\end{KeepFromToc}

%===============================================================================
\chapter{Introdução~\label{chp:intro}}

\section{Motivação e Contexto}

\todo{%
    Contextualizar sobre o papel do Jornalismo Investigativo, ligeiramente
    sobre as dificuldades de jornalistas na obtenção de dados e aí explicitando
    melhor com o caso de Tribunais Regionais.
}

\section{Objetivo e Escopo}

\todo{%
    Adicionar objetivo e escopo.
}

\section{Método de Pesquisa}

\todo{Adicionar método de pesquisa (provavelmente não será "pesquisa").}


\chapter{Trabalhos Correlatos~\label{chp:related-work}}


\chapter{Descrição das Ferramentas~\label{chp:descrição-das-ferramentas}}

\todo{%
    Esta seção deve ser uma descrição das ferramentas mais relevantes que foram
    \textit{utilizadas} no trabalho, i.e. as dependências (bibliotecas,
    linguagem de programação).
}
\begin{todolist}
    \item Scrapy
    \item SQLite
    \item \ldots
\end{todolist}

\textcolor{blue}{--->> O que estás propondo para descrição de ferramentas (Scrapy e SQLLite) será um pequeno parágrafo de uma Seção chamada "Ferramentas utilizadas" no capítulo de construção da ferramenta. Não vais descrever nenhuma delas com detalhes, pois são ferramentas simplesmente usadas para contruíres teu trabalho. Isso não se detalha porque não são "conceitos". Ferramentas mudam de versão, saem de linha, ficam obsoletas, perdem valor, etc, portanto, não se descrevem em detalhes.}

\chapter{Construção da Ferramenta~\label{chp:construção-da-ferramenta}}

A ferramenta construída neste trabalho, nomeada \textit{\textbf{tjscraper}},
foi produzida para ser capaz de extrair com qualidade
satisfatória~\todo{(Definir ``qualidade satisfatória'' --- em resumo: um
formato que dê de usar os dados para alguma coisa e que não seja ``grego'' para
alguém que não seja da área)} os dados especificamente do TJ do Rio de Janeiro
(TJ-RJ). As estratégias empregadas, portanto, levam em conta acessos que todos
os acessos serão ao TJ-RJ, ainda que visando a possibilidade de serem aplicadas
às páginas dos TJs dos demais estados.

\section{Ferramentas utilizadas~\label{section:ferramentas-utilizadas}}

\textcolor{blue}{--->> O que estás propondo para descrição de ferramentas
(Scrapy e SQLLite) será um pequeno parágrafo de uma Seção chamada "Ferramentas
utilizadas" no capítulo de construção da ferramenta. Não vais descrever nenhuma
delas com detalhes, pois são ferramentas simplesmente usadas para contruíres
teu trabalho. Isso não se detalha porque não são "conceitos". Ferramentas mudam
de versão, saem de linha, ficam obsoletas, perdem valor, etc, portanto, não se
descrevem em detalhes.}

\todo{%
    Esta seção deve ser uma descrição das ferramentas mais relevantes que foram
    \textit{utilizadas} no trabalho, i.e. as dependências (bibliotecas,
    linguagem de programação).
}
\begin{todolist}
    \item Scrapy
    \item SQLite
    \item \ldots
\end{todolist}

\section{Organização geral}

\begin{todolist}
    \item Explicar a construção da ferramenta como CLI, biblioteca e como
          aplicação Web;
    \item Diagramar módulos;
\end{todolist}

A ferramenta foi projetada como uma biblioteca de \textit{software} acompanhada
de duas interfaces de usuário~\footnote{Neste ponto, ``usuário'' refere-se a
alguém que não esteja utilizando a ferramenta como uma biblioteca de código, e
sim como aplicação.}: uma aplicação de Interface de Linha de Comando
(\textit{ILC}\footnote{Comumente referenciado com a sigla em inglês ``CLI'', de
``Command Line Interface''.} e uma aplicação de servidor Web.

A ILC tem como objetivo permitir o uso de boa parte dos diferentes recursos da
biblioteca em um terminal para a execução tarefas pontuais, como exibir
informações sobre estado atual da \textit{cache}, iniciar um processo de
extração, ou mesmo iniciar uma instância de servidor Web da ferramenta para
fins de desenvolvimento.

O servidor Web é destinado ao usuário que queira uma interface de simples
acesso e uso da ferramenta, primariamente focando na obtenção de dados dos TJs
com os recursos que não estejam disponíveis nas páginas oficiais dos TJs
(descritos na~\todo{Referenciar Seção}), tendo em mente como público alvo
jornalistas.

\section{Estratégias de extração}

O TJ-RJ possui rotas e subdomínios que dão acesso aos mesmos dados por vias
diferentes. Delas, foram elaboradas estratégias para extrair dados da rota que
os fornece em formato HTML~\footnote{\todo{link da rota HTML}}, conhecida antes
da produção efetiva da ferramenta, e da que os fornece em formato
JSON~\footnote{\todo{link da rota JSON}}, encontrada durante o processo de
produção e portanto implementada posteriormente. O formato HTML é apresentado
como uma página de consulta para um usuário comum, enquanto o formato JSON é
um conjunto de rotas que implementam uma API Web.

A extração, em todos os casos, leva em conta a busca pelo número do processo.
Foi escolhido esse campo de busca por conta do domínio de valores válidos ser
limitado e conhecido: números de processos são dados seguindo os formatos
numéricos fixos unificado e antigo, respectivamente
``NNNNNNN-NN.AAAA.N.NN.NNNN'' e ``AAAA.NNN.NNNNNN-N'' em que ``AAAA'' é o ano
de criação do documento do processo e cada ``N'' é um dígito de 0 a 9, logo é
possível encontrar todos os processos apenas fixando o ano e testando
diferentes combinações para os demais dígitos. Outros campos, como nome das
partes por exemplo, possuem uma abrangência de valores muito grande e difícil
de se conhecer em sua totalidade ou, nos casos como a busca por Sentença, não
são capazes de, mesmo exauridas todas as combinações de valores de entrada, dar
acesso a todos os processos registrados.

\subsection{Estratégia inicial: extração HTML}

\begin{todolist}
    \item Iniciar descrevendo como um usuário comum faz uma busca por número do processo.
    \item Descrever a situação inicial das páginas do TJ-RJ;
    \item Ressaltar dificuldades com páginas diferentes, fornecendo links para
          exemplos --- para fins de reprodutibilidade, o link será uma cópia
          para um .html salvo no repositório do código da ferramenta, na pasta
          de ``samples'' (amostras);
    \item Explicar como Scrapy foi utilizado.
\end{todolist}

Para extração das rotas do TJ-Rj que fornecem seu conteúdo no formato HTML, a
estratégia é semelhante a reproduzir o acesso de um usuário comum ao efetuar
uma busca.

\review{A rota de Consultas Processuais
unificada~\footnote{\url{http://www4.tjrj.jus.br/ConsultaUnificada/consulta.do}}
exibe uma página que permite a busca por processos por ambos os tipos de
numeração.}

% :)

\begin{minted}[autogobble]{html}
    <tr>
      <td class="info" valign="top" nowrap="nowrap">Assunto:</td>
      <td valign="top">
        Crimes de Tortura (Art. 1º - Lei 9.455/97) E
Prevaricação (Art. 319 e 319-A - CP) E Usurpação de função pública
(Art. 328 - CP)
      </td>
    </tr>
\end{minted}

\subsection{Estratégia alternativa: varredura com API JSON}

\begin{todolist}
    \item Iniciar descrevendo a relação entre parâmetros de busca da busca HTML
          e as rotas como \texttt{/por-numero/}.
    \item Expor sobre a existência de múltiplos sub-domínios do TJ-RJ;
    \item Complementar com a falta de documentação da API do TJ-RJ (forma de
          operação descoberta na base de tentativa e erro);
    \item Explicar rapidamente sobre a filtragem dos campos;
\end{todolist}

\section{Estratégias de aceleração de consulta}

\todo{%
    Para cada estratégia, nomeá-la para referenciar na parte de resultados
    demonstrando os ganhos de aceleração. Nesta seção, não tomar conclusões
    sobre ganhos, apenas enunciar e explicar como as estratégias funcionam e
    por que elas foram decididas dessa forma.
}

\todo{%
    Na seção de resultados, comparar um cenário base (sem aplicar as estratégias) com:
}
\begin{todolist}
    \item Apenas uma das estratégias aplicadas (fazer para cada estratégia);
    \item Com todas as estratégias aplicadas.
\end{todolist}
\todo{%
    A hipótese lançada é de que o maior tempo desprendido é com IO (e não
    processamento das requisições).
    A conclusão deverá confirmar essa hipótese demonstrando que com a redução
    (através de caching e através de requisições assíncronas) do gasto com IO o
    tempo de consulta se reduz de forma diretamente proporcional.
}

\subsection{Requisições assíncronas}

\subsection{\textit{Cache} dos resultados}

\begin{todolist}
    \item Justificar a necessidade de \textit{cache} para os resultados: em
          resumo, requisições, sejam de um mesmo usuário ou de usuários
          diferentes, podem muito bem conter uma intersecção dos processos que
          serão buscados.
\end{todolist}

\chapter{Configuração Experimental~\label{chp:configs}}


\chapter{Análises Experimentais~\label{chp:Resultados-Experimentais}}

Nesta seção estão descritos os resultados obtidos a partir de experimentos
visando como objetivo principal averiguar a capacidade da ferramenta de obter e
exportar dados de processos de um TJ em tempo viável. Para isso, adotou-se como
critérios de análise o \textbf{tempo total da consulta} de
um intervalo de processos, separados entre tempo de CPU, IO de Rede e IO de
Armazenamento, e o \textbf{tempo médio} para se descobrir um processo em um
intervalo.

\section{Configuração Experimental~\label{sec:Configuração-Experimental}}

Os experimentos foram realizados em uma estação de trabalho com um Intel(R)
Core(TM) i5-7300HQ CPU @ 2.50GHz, 16GB RAM DDR4 2133MHz e interface de rede
1000Mbps em uma conexão residencial.

Os dados de temporização foram obtidos através do perfilamento de uma chamada
completa do processo de extração via API JSON do TJ-RJ de um intervalo de
processos variando os parâmetros com todas as combinações dos valores
especificados na~\Cref{tbl:parâmetros-de-perfilamento}. Nas combinações com
tamanho do lote em 1, o cenário é classificado como Síncrono (Sync), e os
demais como Assíncrono (Async).

\begin{table}[htb]
  \centering
  \begin{tabular}{ll}
    \toprule
    Parâmetro & Valores \\
    \midrule
    Número inicial do intervalo & 15712 \\
    Tamanho do intervalo do campo ``NNNNNNN'' (nº de processos) & 10, 50, 100, 1000 \\
    Tamanho do lote & 1, 10, 100, 500, 1000 \\
    \bottomrule
  \end{tabular}
  \caption{%
    Valores utilizados para os parâmetros de extração de dados de processos.
  }
  \label{tbl:parâmetros-de-perfilamento}
\end{table}

\section{Resultados Experimentais}

A~\Cref{gra:tempos-async-vs-sync} mostra um comparativo do tempo decorrido para
uma implementação síncrona e uma assíncrona nos diferentes cenários de
configuração, com a~\Cref{tbl:tempos-async-vs-sync} expondo o tempo por
processo do mesmo comparativo, obtido dividindo-se o tempo total naquela
configuração pelo número de processos. Percebe-se por esses dados uma redução
significativa no tempo de IO de Rede (de 66\% a 98\%) mantendo um tempo de CPU
bastante próximo, demonstrando a eficiência do uso de requisições assíncronas.

É possível perceber que o uso de requisições assíncronas traz uma maior redução
no tempo médio com uma quantidade maior de processos, uma vez que um conjunto
de poucos processos tira pouco proveito da assincronia não substituindo o tempo
de ociosidade com o tempo de se disparar mais requisições, enquanto um número
maior é capaz de preencher esse tempo.

\begin{figure}[htb]
    \centering
    \begin{tikzpicture}
        \pgfplotstableread{io_stats-async-sync.csv}{\table}
        \pgfplotstablegetcolsof{\table}
        \pgfmathtruncatemacro\numberofcols{\pgfplotsretval-1}
        \pgfplotstablegetcolumnnamebyindex{0}\of{\table}\to{\colprincipal}
        \begin{axis}[
            ybar,
            enlarge x limits=0.15,
            symbolic x coords={
              10, 50, 100, 1000,
            },
            xtick=data,
            xticklabels from table={\table}{\colprincipal},
            ylabel={Tempo (s)},
            xlabel={Nº de processos},
            legend cell align=left,
            legend style={
                legend pos=outer north east,
                cells={align=left},
            },
            width=0.7\textwidth,
            height=14em,
            clip=false,
        ]
            \pgfplotsinvokeforeach{1,...,\numberofcols}{
                \pgfplotstablegetcolumnnamebyindex{#1}\of{\table}\to{\colname}
                \addplot table [y index=#1] {\table};
                \addlegendentryexpanded{\colname}
            }
        \end{axis}
    \end{tikzpicture}
    \caption{%
      Comparativo de tempo desprendido com IO de Rede e processamento em CPU
      entre o cenário sem a aplicação de programação assíncrona via IO
      não-bloqueante (Sync) e com a aplicação (Async).
    }
    \label{gra:tempos-async-vs-sync}
\end{figure}

\begin{table}[htb]
  \centering
  \begin{tabular}{lllll}
    \toprule
     & \multicolumn{2}{c}{Sync} & \multicolumn{2}{c}{Async} \\
    Nº de processos & IO de Rede & CPU & IO de Rede & CPU \\
    \midrule
    10 & 0.1850 & 0.0039 & 0.0158 & 0.0020 \\
    50 & 0.2386 & 0.0104 & 0.0022 & 0.0026 \\
    100 & 0.2021 & 0.0106 & 0.0012 & 0.0030 \\
    1000 & 0.2272 & 0.0061 & 0.0008 & 0.0032 \\
    \bottomrule
  \end{tabular}
  \caption{%
    Tempo médio de busca por processo comparando os cenários Síncrono e
    Assíncrono.
  }
  \label{tbl:tempos-async-vs-sync}
\end{table}

Quanto à separação das requisições em lotes, as medições de temporização estão
expostas na~\Cref{gra:tempos-tamanhos-de-passo-async}. Os resultados para um
intervalo 10000 processos estão separados
na~\Cref{gra:tempos-tamanhos-de-passo-async-10000-processos} para melhor
visualização. De maneira geral, tamanhos de lote maiores ou iguais que 100
tiveram resultados similares para quase todos os tamanhos de intervalo de
processos, tendo como excepcionalidade o intervalo de 10 processos com tamanho
do lote em 500. Essa excepcionalidade, porém serve apenas para uma observação
comparativa, visto que TJs possuem na ordem de centenas de milhares de
processos por ano.

Para analisar essas medidas é preciso levar em conta que, para evitar sanções e
sem acesso a um \texttt{robots.txt}, não é possível ter sempre o tamanho do
lote igual ao do intervalo de processos. Partindo dessa premissa, as medidas
expõem que, ao menos para um intervalo de até 1000 processos com tamanhos de
lote maiores ou iguais a 100, não há ganhos significativos em se manter o
tamanho de lote tão próximo do tamanho do intervalo, porém para os casos de
intervalos de tamanho 1000 e 10000 os lotes de tamanho 500 se demonstraram
suficientes e com os melhores resultados.

\begin{figure}[htb]
    \centering
    \begin{tikzpicture}
        \pgfplotstableread{io_stats-batch-size.csv}{\table}
        \pgfplotstablegetcolsof{\table}
        \pgfmathtruncatemacro\numberofcols{\pgfplotsretval-1}
        \pgfplotstablegetcolumnnamebyindex{0}\of{\table}\to{\colprincipal}
        \begin{axis}[
            ybar,
            enlarge x limits=0.15,
            symbolic x coords={
                10, 100, 500, 1000
            },
            ylabel={Tempo (s)},
            xlabel={Tamanho do lote},
            legend cell align=left,
            legend style={
                legend pos=outer north east,
                cells={align=left},
            },
            xtick=data,
            xticklabels from table={\table}{\colprincipal},
            width=0.7\textwidth,
            clip=false,
        ]
            \pgfplotsinvokeforeach{1,...,\numberofcols}{
                \pgfplotstablegetcolumnnamebyindex{#1}\of{\table}\to{\colname}
                \addplot table [y index=#1] {\table};
                \addlegendentryexpanded{\colname}
            }
        \end{axis}
    \end{tikzpicture}
    \caption{%
        Comparação do tempo de obtenção de todos os processos de diferentes
        intervalos conforme tamanho do lote.
    }
    \label{gra:tempos-tamanhos-de-passo-async}
\end{figure}

Analisando-se o tempo de espera por IO de Rede
na~\Cref{gra:tempos-tamanhos-de-passo-async-10000-processos}, dado em uma média
aritmética por processo, percebe-se também que um tamanho de lote muito grande
(nesse caso, 1000) acarreta em um aumento no tempo de espera, o que deve
explicar o fato de que o tempo total de consulta passa a aumentar com relação
aos lotes de tamanho 500.

\begin{figure}[htb]
    \centering
    \begin{tikzpicture}
        \pgfplotstableread{io_stats-batch-size-10000.csv}{\table}
        \pgfplotstablegetcolsof{\table}
        \pgfmathtruncatemacro\numberofcols{\pgfplotsretval-1}
        \pgfplotstablegetcolumnnamebyindex{0}\of{\table}\to{\colprincipal}
        \begin{axis}[
            ybar,
            enlarge x limits=0.15,
            symbolic x coords={
                100, 500, 1000
            },
            bar shift=-0.5em,
            ylabel={Tempo total (s)},
            xlabel={Tamanho do lote},
            axis y line*=left,
            xtick=data,
            ymin=0,
            xticklabels from table={\table}{\colprincipal},
            width=0.7\textwidth,
            height=14em,
            clip=false,
            %legend style={at={(1,1)},anchor=north west},
            legend style={at={(0,1.1)},anchor=south west},
        ]
            \pgfplotsinvokeforeach{1,...,\numberofcols}{
                \pgfplotstablegetcolumnnamebyindex{#1}\of{\table}\to{\colname}
                \addplot table [y index=#1] {\table};
                \legend{Tempo total};
            }
        \end{axis}
        \begin{axis}[
            ybar,
            enlarge x limits=0.15,
            symbolic x coords={
                100, 500, 1000
            },
            bar shift=0.5em,
            axis y line*=right,
            ylabel={Tempo (IO de Rede) (s)},
            xlabel={Tamanho do lote},
            xtick=data,
            ymin=0,
            xticklabels from table={\table}{\colprincipal},
            width=0.7\textwidth,
            height=14em,
            clip=false,
            legend style={at={(1,1.1)},anchor=south east},
        ]
          \addplot [
            red,
            fill=red!35,
          ] coordinates {
            (100, 1.27)
            (500, 1.02)
            (1000, 4.47)
          };
          \legend{Tempo de espera por IO de Rede};
        \end{axis}
    \end{tikzpicture}
    \caption{%
        Comparação do tempo de obtenção de todos os processos de um intervalo
        de tamanho 10000 conforme tamanho do lote.
    }
    \label{gra:tempos-tamanhos-de-passo-async-10000-processos}
\end{figure}

Para medir a eficiência da \textit{cache}, foram separados três intervalos de
valores para ``NNNNNNN'' com intersecção, $I_1 = [15712, 16712]$, $I_2 =
[16212, 17212]$ e $I_3 = I_1 \cup I_2$, e 3 cenários
diferentes~\Cref{tab:descrição-cenários-cache} com os mesmos parâmetros de
consulta (com exceção do intervalo de processos). As medidas de eficiência,
dadas em termos do tempo total de execução e do tempo gasto com IO de Rede
(ambos em segundos) estão expostas na~\Cref{gra:tempo-cache}. A execução de
cada cenário se dava com uma \textit{cache} inicialmente vazia.

No cenário com intersecção completa (cenário $C$), como esperado, na consulta
$C_2$ o tempo de IO de Rede é zerado visto que todos os processos que $C_2$
baixaria já foram previamente salvos na \textit{cache} por $C_1$ e portanto
sendo carregados dela em vez da rede. É interessante notar que, com isso, o
tempo total da consulta também se reduziu drasticamente, demonstrando que o uso
da \textit{cache} foi compensatório. No cenário com intersecção parcial
(cenário $A$), novamente o tempo total de consulta também se reduziu
significativamente além do tempo de IO de Rede. Comparando as consultas $A_2$ e
$B$ (que operam sob o mesmo intervalo, $I_2$), reforçam-se as conclusões
analisadas em $C$: para um mesmo intervalo, a \textit{cache} ainda que parcial
foi capaz de reduzir ambos o tempo de IO de rede e o tempo total da consulta,
sendo novamente compensatória.

\begin{table}[htb]
  \begin{tabular}{cp{0.4\textwidth}p{0.4\textwidth}}
    \toprule
    Cenário & Objetivo & Características \\
    \midrule
    A
    &
    Medir a eficiência da cache em reduzir parte do retrabalho.
    &
    Separado em duas consultas feitas em sequência: $A_1$ (intervalo $I_1$) e
    $A_2$ (intervalo $I_2$). A \textit{cache} preenchida por $A_1$ é
    reutilizada para $A_2$.
    \\
    B & Medir o tempo de $A_2$ sem \textit{cache}. & Uma única consulta no intervalo $I_2$.
    \\
    C
    &
    Averiguar se há compensação em utilizar a \textit{cache} para reduzir o
    tempo de busca em caso de retrabalho.
    &
    Separado em duas consultas, $C_1$ e $C_2$, no intervalo $I_3$
    reaproveitando a \textit{cache} preenchida por $C_1$ em $C_2$.
    \\
    \bottomrule
  \end{tabular}
  \caption{Descrição dos cenários utilizados para medição da eficiência da \textit{cache}.}
  \label{tab:descrição-cenários-cache}
\end{table}

\begin{figure}[htb]
    \centering
    \begin{tikzpicture}
        \pgfplotstableread{io_stats-cache-effect.csv}{\table}
        \pgfplotstablegetcolsof{\table}
        \pgfmathtruncatemacro\numberofcols{\pgfplotsretval-1}
        \pgfplotstablegetcolumnnamebyindex{0}\of{\table}\to{\colprincipal}
        \begin{axis}[
            ybar,
            bar shift=-0.5em,
            enlarge x limits=0.15,
            symbolic x coords={
                A1, A2, B, C1, C2,
            },
            axis y line*=left,
            ylabel={Tempo total (s)},
            xlabel={Consulta},
            legend cell align=left,
            legend style={
              at={(0,1.1)},
              anchor=south west,
            },
            xtick=data,
            xticklabels from table={\table}{\colprincipal},
            ymin=0,
            width=0.7\textwidth,
            clip=false,
        ]
            \pgfplotstablegetcolumnnamebyindex{1}\of{\table}\to{\colname}
            \addplot table [y index=1] {\table};
            \addlegendentryexpanded{\colname}
        \end{axis}
        \begin{axis}[
            ybar,
            bar shift=0.5em,
            enlarge x limits=0.15,
            symbolic x coords={
                A1, A2, B, C1, C2,
            },
            axis y line*=right,
            ylabel={Tempo de espera por IO de Rede (s)},
            legend cell align=left,
            legend style={
              at={(1,1.1)},
              anchor=south east,
            },
            xtick=data,
            xticklabels from table={\table}{\colprincipal},
            ymin=0,
            width=0.7\textwidth,
            clip=false,
        ]
            \pgfplotstablegetcolumnnamebyindex{2}\of{\table}\to{\colname}
            \addplot [red, fill=red!35] table [y index=2] {\table};
            \addlegendentryexpanded{\colname}
        \end{axis}
    \end{tikzpicture}
    \caption{%
      Comparação do tempo de obtenção de todos os processos em 5 consultas
      específicas a fim de demonstrar a eficiência do uso de \textit{cache}.
    }
    \label{gra:tempos-cache}
\end{figure}

\chapter{Análise de Usabilidade~\label{chp:Análise-de-Usabilidade}}

\begin{todolist}
    \item Expor a ferramenta (versão web) para o pessoal do JI;
    \item Comentar sobre a impressão que tiveram do uso da ferramenta;
    \item Criar um formulário com perguntas para eles poderia ser bom.
\end{todolist}

\chapter{Conclusões e perspectivas~\label{chp:conclusions}}


\todo{Conclusões da parte dos TJs:}
\begin{todolist}
    \item Comentar o impacto da informatização de um sistema para além do
          público-alvo direto do sistema (ex: quem se beneficia não é somente
          um juíz, procurador, etc., mas também jornalistas que precisam de
          estatísticas relacionadas aos processos).
    \item Comentar sobre o impacto de decisões de formatos de distribuição e
          APIs no processo de informatização.
\end{todolist}

\todo{Conclusões da parte das bibliotecas:}
\begin{todolist}
    \item Comentar da dificuldade de testagem
\end{todolist}


%-------------------------------------------------------------------------------

\end{document}
